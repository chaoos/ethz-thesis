\chapter{\texorpdfstring{openQ$^\star$D}{openQxD}}
\label{ch:p1:openqxd}

\newcommand{\Dw}{{D_{\mathrm{w}}}}

The openQxD code~\cite{openqxd} is an extension of the state-of-the-art
lattice QCD Markov Chain Monte Carlo code
openQCD~\cite{openqcd,Luscher:2012av}, which is written in the \worktodo{C89} standard
with MPI parallelization with proven strong-scalability to hundreds of
CPU-based nodes and specific optimizations for x86 architectures.
Due to the extensive suite of legacy applications based on openQCD, as well as
its renowned stability and reproducibility, our goal is to accelerate the code
by offloading the solution of a large linear system of equations
$\Dw\psi=\eta$, defined by the Dirac operator $\Dw$, to the
device using the QUDA library.
The solution of this system for many right-hand sides is the most time
intensive kernel of the Monte Carlo algorithm, whose correctness can be
checked a posteriori, thereby allowing us to retain the functionality of
existing applications while interfacing only a minimal set of parameters.

The Wilson-Clover Dirac operator $\Dw$ is a nearest-neighbor stencil operator
acting on a linear space of quark (or spinor) fields defined on a regular 4D
cubic lattice of $V_\mathrm{G}=N_0N_1N_2N_3$ sites, typically in the range $10^6-10^8$.
Due to the nearest-neighbor coupling, a good parallelization is achieved by
domain decomposition, where a contiguous local volume of size $V_\mathrm{L}=L_0L_1L_2L_3$,
which divides the global problem size, is associated with a single computing
unit.
With accessible problem sizes, the Dirac operator is poorly conditioned and
multi-grid preconditioning such as the inexact deflation \cite{Lüscher2007deflation} provided in openQCD
or its QUDA equivalent is needed to speed up the convergence of the iterative
solvers used.

The fine-grid problem which we wish to offload is defined by the right-hand
side spinor $\eta$, which may be represented in different ways depending on 
conventions. These different conventions need to be mapped into each other in the interface.
Likewise, the couplings between the sites, which define the Dirac operator, are
parameterized by a set of so-called vector (or gauge) field variables $U_\mu$
for the four directions $\mu=0,1,2,3$, which need to be passed through the interface as well.
Although these fields are periodically updated along the Markov chain, in
the measurement part of the workflow they are kept fixed.
Finally, it is convenient to precompute the site-local coupling called the
clover-field which is a function of the gauge-field variables.
All these parameters are represented in openQCD by arrays of structs of length of
the local lattice volume which are described as follows.

The double precision spinor field is stored as an array of $V_\mathrm{L}$ \code{spinor\_dble} structs (\cref{lst:spinor}), the gauge field is stored as an array of $4V_\mathrm{L}$ \code{su3\_dble} structs (\cref{lst:gauge}), whereas the clover field is an array of $2V_\mathrm{L}$ \code{pauli\_dble} (\cref{lst:clover}).

The gauge field is stored as an array of $4V_\mathrm{L}$ \code{su3\_dble} structs (\cref{lst:gauge}), which rely on \code{complex\_dble} structs (\cref{lst:complex_double}). The double precision spinor field is stored as an array of $V_\mathrm{L}$ \code{spinor\_dble} structs (\cref{lst:spinor}), while the clover field is an array of $2V_\mathrm{L}$ \code{pauli\_dble} (\cref{lst:clover}). 

\begin{codelisting}
\begin{minted}[]{cpp}
typedef struct
{
   double re,im;
} complex_dble;
\end{minted}
\caption{The complex double struct}
\label{lst:complex_double}
\end{codelisting}

\begin{codelisting}
\begin{minted}[]{cpp}
typedef struct
{
   complex_dble c11,c12,c13,c21,c22,c23,c31,c32,c33;
} su3_dble;
\end{minted}
\caption{The gauge field struct}
\label{lst:gauge}
\end{codelisting}


%\todo{Since the struct complex double is present in the spinor and gauge-field structs, I believe it is better for the listing to be split.}
% typedef struct
% {
%    double re,im;
% } complex_dble;

\begin{codelisting}
\begin{minted}[]{cpp}
typedef struct
{
   complex_dble c1,c2,c3;
} su3_vector_dble;

typedef struct
{
   su3_vector_dble c1,c2,c3,c4;
} spinor_dble;
\end{minted}
\caption{The spinor field struct}
\label{lst:spinor}
\end{codelisting}


\begin{codelisting}
\begin{minted}[]{cpp}
typedef struct
{
   double u[36];
} pauli_dble;
\end{minted}
\caption{The clover field struct}
\label{lst:clover}
\end{codelisting}

The spinor field has indices $(x, \alpha, a)$ ($a$ running fastest), where $x=(x_0,x_1,x_2,x_3)$ is the lattice index (a Euclidean 4-vector), $\alpha \in \{0,1,2,3\}$ is the spinor index and $a \in \{0,1,2\}$ is the color index.

The gauge field instead has indices $(x_{odd}, \pm \mu, a, b)$, where $x_{odd}$ is the lattice index (only odd points, $V_\mathrm{L}/2$ elements), $\mu \in \{\pm 0, \pm 1, \pm 2, \pm 3\}$ is the direction of the gauge link (in positive as well as negative direction, i.e., 8 possibilities) and $a,b$ parameterize the row and column of the $SU(3)$-valued gauge link (row-major). Thus openQxD stores the gauge links in all $8$ directions at odd lattice points in contrast with most other common conventions which store $4$ gauge links in positive directions for every lattice point.

Finally, the clover field has indices $(x, \pm, i)$, where $x$ is the lattice index, $\pm \in \{+, -\}$ denotes the chirality (i.e. the upper ($+$) or lower ($-$) $6 \times 6$-block of the $12 \times 12$ clover matrix) and $i \in \{0, \dots, 35\}$ the $36$ non-zero real numbers needed to parametrize a $6 \times 6$ Hermitian matrix.

In addition to these data structs, already defined in openQCD, the openQxD package --
which provides an extension to incorporate electromagnetic gauge fields --
requires new functionality such as C$^\star$ boundary conditions
and additional degrees of freedom: these are described in more detail in
\worktodo{sec:qcd+qed,sub:cstar} and respectively.
