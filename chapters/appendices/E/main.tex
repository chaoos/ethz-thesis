\chapter{Appendix: Conventions and Notation}
\label{ch:appendix:notation}

\section{Frequently used names}%
\vskip -2em
\begin{tabularx}{\textwidth}{lX}
  %\toprule%
  %\tableheadline{Symbol} & \tableheadline{Meaning} \\
  %\midrule%
  openQCD & A framework for lattice QCD simulations on CPUs. \\
  openQxD & A framework for lattice QCD+QED simulations on CPUs based on openQCD v1.6. \\
  QUDA & A library for performing calculations in lattice QCD on GPUs. \\
  %\bottomrule
\end{tabularx}

\section{Physical constants}
\label{sec:physical:constants}
\sisetup{separate-uncertainty=false}
\vskip -2em
\begin{tabularx}{\textwidth}{lX}
  $\hbar$  & reduced Planck constant, $\hbar=\u{1.054571817d-34}{\joule\second}$ \\
  $e$      & elementary electric charge, $e=\u{1.602176634d-19}{\coulomb}$ \\
  $\alpha$ & fine-structure constant, $\alpha=\u{0.0072973525643}{}$ \\
\end{tabularx}
\begin{flushright}
\cite{codata:2022}
\end{flushright}
\sisetup{separate-uncertainty=true}

\section{Frequently used symbols and conventions}%
% \vskip -2em
% \begin{tabularx}{\textwidth}{lX}
%   %\toprule%
%   %\tableheadline{Symbol} & \tableheadline{Meaning} \\
%   %\midrule%
%   $E$ & energy \\
%   $m$ & rest mass \\
%   $p$ & momentum \\
%   %\bottomrule
% \end{tabularx}

\paragraph{Units.}
We shall work with natural units $\hbar = c = 1$.

\paragraph{Spinor product.}
%The spinor product is denoted by $\sprod{\cdot}{\cdot}$ and is the obvious one where by convention the first component is conjugated.
$\sprod{\cdot}{\cdot}$ denotes the standard spinor product where by convention the first component is conjugated.

\paragraph{Spinor norm.}
The spinor norm is defined to be the one induced by the spinor product, $\norm{\psi} = \sqrt{\sprod{\psi}{\psi}}$.

\paragraph{Hermitian transpose.}
The conjugate transpose of a linear operator $A$ or a spinor $\psi$ is denoted with the dagger symbol $A^\dagger$, $\psi^\dagger$ and usually applies to all indices unless stated otherwise.

\paragraph{Dirac adjoint.}
The Dirac adjoint of a spinor $\psi$ is denoted by $\bar{\psi}$ and defined as $\bar{\psi} = \psi^\dagger i \gamma^0$.

\paragraph{Set cardinality.}
The cardinality of a finite set $\mathcal{A}$ is denoted by $\lvert \mathcal{A} \rvert$ and coincides with the natural number found by counting its elements.

\section{Index notation}
\label{sec:notation:index}

A vector $v$ may have multiple indices associated to it.
With only one index unless stated otherwise, the $i$-th component of the vector $v$ will be denoted with a special subscript resembling array-notation as $v_{[i]}$.
On the other hand, a regular super- or subscript as $v_i$ or $v^{i}$ will denote the $i$-th element of a list of vectors.
By this we can easily write $v_{i [j]}$ denoting the $j$-th \emph{component} of the $i$-th \emph{vector}.

\paragraph{Spacetime indices.}
Spacetime indices are denoted with Latin letters starting from $x, y, z, w, \ldots$

\paragraph{Spinor indices.}
Spinor indices are denoted with Greek letters starting from $\alpha, \beta, \ldots$

\paragraph{Lorentz indices.}
Lorentz indices are denoted with Greek letters starting from $\mu, \nu, \rho, \sigma, \ldots$

\paragraph{Color indices.}
Color indices are denoted with Latin letters starting from $a, b, c, \ldots$

An example carrying multiple indices is the propagator $S$, for instance
\begin{equation}
\propxyab{x}{y}{\alpha}{\beta}{a}{b} \;.
\end{equation}


\section{\texorpdfstring{$\gamma$}{Gamma}-matrix basis}

We work in the Weyl basis,
\begin{align} \label{eq:gamma:weyl:basis}
\gamma_0 &=
\begin{pmatrix}
0 & 0 & 1 & 0 \\
0 & 0 & 0 & 1 \\
1 & 0 & 0 & 0 \\
0 & 1 & 0 & 0
\end{pmatrix} \;,
&
\gamma_1 &=
\begin{pmatrix}
0 & 0 & 0 & 1 \\
0 & 0 & 1 & 0 \\
0 & -1 & 0 & 0 \\
-1 & 0 & 0 & 0
\end{pmatrix} \;, \\
\gamma_2 &=
\begin{pmatrix}
0 & 0 & 0 & -i \\
0 & 0 & i & 0 \\
0 & -i & 0 & 0 \\
i & 0 & 0 & 0
\end{pmatrix} \;,
&
\gamma_3 &=
\begin{pmatrix}
0 & 0 & 1 & 0 \\
0 & 0 & 0 & -1 \\
-1 & 0 & 0 & 0 \\
0 & 1 & 0 & 0
\end{pmatrix} \;,
\end{align}
but all conclusions drawn in this document hold for other choices too.

