\chapter{Introduction}
\label{ch:introduction}

\dictum[Dante Alighieri]{%
  Abandon all hope, ye who enter here. }%
\vskip 1em

\readit{1}

\worktodo{
   * we focus on these publications (mg-lma, GPU stuff, not cstar HVP stuff)
   * motivation (1-2 pages)
   * intro in lattice field theory (fundamentals)
   * action, O(a) improvement
   * Cstar, QCD+QED
   * Dirac operator; already done in p2 intro
   * fix the physical point
   * fix the scale, 
   * recover the continuum limit
* GPUs
* noise reduction methods
   * scope of this work
   * Structure of the thesis: Part 1 deal with the GPU stuff, part2 the algorithmic improvement noise reduction and stuff
   * overall 5-10 pages
   * topics: lattice field theory intro, dirac op, QCD+QED, C*, (first 2 pages of GPU-intro), (first page g-2 intro), Research problem and objectives, Summary of Contributions, non-perturbative treatment necessary, only ab initio method for non-perturbative access to QCD, Structure of the thesis
}

The standard model of particle physics is one of the most successful frameworks for theoretical predictions of physical quantities.
Its agreement with experimental data is unprecedented.
Lattice field theory (LFT) it still the only regularization of quantum field theory (QFT) and by this the only ab initio method that allows non-perturbative access to observables.
Due to its statistical nature, determinations from lattice field theory can be systematically improved.
Gold-plated observables like the anomalous magnetic momentum of Muon $g-2$, where low energy behavior of QCD plays an important role are accessible only non-perturbatively with the application of lattice QCD and QED.
The fields of the continuum theory are discretized in spacetime and represented as degrees of freedom associated to sites and links in a four dimensional spacetime hypercube.
The discretization of the theory naturally regularizes infrared and ultraviolet divergences occurring in the continuum theory, due to its finite resolution and volume description.
Feynman's path integral description of QFT is used as a backend, where the functional integral over field degrees of freedom turns into a high-dimensional regular integral.
Dimensionality may go up to $10^{8}$ which introduces the need of modern high-performance computing (HPC) infrastructure for their evaluation.
Careful understanding of systematic and statistical errors allows studying physical quantities of interest and thorough comparison to experiment.
Discretizing a theory comes with many decisions and choices; all with different advantages and disadvantages.
Independent research groups in the field, working with different discretizations regularly compare their results and discuss systematics.
Once extrapolated to the continuum lattice results should show independence of these choices and agree within their error estimates.

\section{Fundamentals of lattice field theory}

As already touched upon, lattice field theory is a discretized version of continuum quantum field theory.
As introduced by Feynman in 1948~\cite{Feynman1948} expectation values of observables in context of continuum quantum field theory can be mathematically modeled as functional integrals over the involved fundamental fields
\begin{equation} \label{eq:path:integral:expectation:value}
\langle \observable{O}[\psi, \bar{\psi}, A] \rangle =
\frac{1}{Z} \int
\DD \psi
\DD \bar{\psi}
\DD A \;
\observable{O}[\psi, \bar{\psi}, A]
e^{i S[\psi, \bar{\psi}, A]} \;.
\end{equation}
In this equation, the fundamental degrees of freedom are the fields $\psi$, $\bar{\psi}$ and $A$.
$S$ denotes the action of the theory and $Z = \int \DD \psi \DD \bar{\psi} \DD A e^{i S[\psi, \bar{\psi}, A]}$ the partition function.
The observable $\observable{O}$ is a functional of the fields and can be any time-ordered correlation function.
Here $\psi$ and $\bar{\psi}$ may represent fermionic particles and $A$ a gauge boson.
The bosonic field is vectorial and the fermionic one spinorial and Grassmann-valued.
It is a standard result in fermionic QFT that the fermion degrees of freedom in the partition function $Z$ can be integrated out, such that
\begin{equation}
Z = \int \DD A \det(D) e^{i S[\psi, \bar{\psi}, A]} \;.
\end{equation}
The non-locality of the determinant encodes all possible paths of fermion represented by $\psi$.
We encounter for the first time the operator that has a central role in this document; the Dirac operator $D$, here in its continuum form $D = i \gamma^{\mu} D_{\mu} - m$ with the Dirac matrices obeying the Clifford algebra $\{\gamma^{\mu}, \gamma^{\nu}\} = 2 \eta^{\mu \nu}$, where $\eta$ is the Minkowski spacetime metric\footnote{In particle physics notation usually of signature $(+---)$.}, $m$ is the fermion mass and $D_{\mu}$ are covariant derivatives.
These derivatives were introduced to retain Lorentz-invariance of the kinetic fermion part of the action $S$.
Inevitably this introduced another field $A$ which is vector-valued, massless and mediates the interaction among fermions.
In the context of QED, this results in the photon mediating the electromagnetic interaction, whereas for an $\ggrp{SU}{3}$ gauge group this results in the 8 gluons mediating the strong interaction of quantum chromodynamics (QCD).

In order to apply a classical statistical model to the evaluation of \cref{eq:path:integral:expectation:value}, we have to perform a Wick-rotation translating Minkowski into Euclidean spacetime, $t_m \to -i t_e$, where $t_m$ is the real Minkowski time and $t_e$ is the (real valued) Euclidean time.
The effect of this on the action is that $e^{i S_m}$ turns into $e^{- S_e}$ allowing the interpretation of the Euclidean action $S_e$ as a Boltzmann weight in a statistical sense as opposed to the Minkowski action $S_m$.
Euclidean correlation functions can be analytically continued to Minkowski space preserving physical contents of the theory~\cite{Osterwalder:1974tc}.
Due to this rotation, correlation functions on the lattice $C(t_e)$, have to be interpreted with Euclidean time.
Euclidean time evolution is thus exponentially decaying as opposed to oscillatory in Minkowski space,
\begin{align}
C(t_e) &= \sum_n A_n e^{ - E_n t_e} \;, &\makebox[\widthof{\text{(Minkowski)}}][l]{\text{(Euclidean)}} \\
C(t_m) &= \sum_n A_n e^{-i E_n t_m} \;. &\text{(Minkowski)}
\end{align}

The Euclidean theory can be discretized by introducing a finite four dimensional spacetime lattice $\Lambda$ of Cartesian coordinates.
Lattice sites are evenly spaced by the dimension-full lattice spacing $a$ such that the coordinates are dimensionless.
We write fields defined on the lattice in resemblance of their continuum analogues as $\psi(x)$ with $x=an$, $n \in \Lambda$ a non-negative integer and dimensionless.
The lattice spacing $a$ provides the finite resolution of the discretization curing IR-divergences with a momentum cutoff proportional to $1/a$ and the finite lattice volume $V = \lvert \Lambda \rvert$ cures UV divergences with a cutoff proportional of $a$.
This provides a natural regularization of the continuum theory.

Fermion fields are defined on the lattice sites with possible further \emph{internal} degrees of freedom for spin and color, whereas gauge fields acting as parallel transporters linking neighboring lattice sites are imagined to live in between sites.
Depending on the gauge group, gauge links possess further internal degrees of freedom with values drawn from the gauge group.

The dynamics of the theory are governed by the continuum Yang-Mills action~\cite{yangmills1954}
\begin{equation} \label{eq:continuum:YM:action}
S = \int \dd x \; \mathcal{L}(x) \;,
\quad
\mathcal{L}(x) = - \frac{1}{4} F^{a}_{\mu \nu}(x) F_{a}^{\mu \nu}(x) + \bar{\psi}(x) (D \psi)(x) \;,
\end{equation}
with the Dirac operator $D$ and the Yang-Mills field strength tensor
\begin{equation}
F_{\mu \nu}^{a}(x) = \partial_{\mu} A_{\nu}^{a}(x) -\partial_{\nu} A_{\mu}^{a}(x) + g f^{abc} A_{\mu}^{b}(x) A_{\nu}^{c}(x) \;.
\end{equation}
The coupling constant $g$ is a free parameter of the theory and the structure constants $f^{abc}$ parametrize the Lie-algebra of the gauge group.

\subsection{Dirac operator}

Dicretizing the Yang-Mills action \cref{eq:continuum:YM:action} and replacing derivatives with symmetric differences results in
\begin{align}
S &= S_G + S_F \;, \\
S_G &= \frac{1}{g^2} \sum_{x \in a \Lambda} \sum_{\mu, \nu} \Re \tr \left[ \delta_{\mu \nu} \id - U_{\mu \nu}(x) \right] \;, \\
S_F &= a^4 \sum_{x \in a \Lambda} \bar{\psi}(x) (D_{\text{naive}}[U] \psi)(x) \;.
\end{align}
The gauge part of the action $S_G$ turns into a sum over plaquettes
\begin{equation}
U_{\mu \nu}(x)
= U_{\mu}(x) U_{\nu}(x + \hat{\mu}) U_{\mu}(x + \hat{\nu})^{\dagger} U_{\nu}(x)^{\dagger} \;,
\end{equation}
where $U_{\mu}$ is the Lie-group-valued gauge field
\begin{equation} \label{eq:gauge:field}
U_{\mu}(x)
= P \exp(i g \int_x^{x + a \hat{\mu}} \dd z_{\mu} A_{\mu}(z) )
\approx e^{i a A_{\mu}(x + \frac{a}{2} \hat{\mu})}
\end{equation}
as exponential of the Lie-algebra-valued field $A$ and $\hat{\mu}$ is the unit 4-vector in direction $\mu$.
As constructed \cref{eq:gauge:field}, gauge links are explicitly defined in between lattice sites.
The fermion part $S_F$ introduces the discretized Dirac operator -- the naive lattice Dirac operator
\begin{equation}
% (D \psi)(x) = (4+m)\psi(x) \\
% + \frac{1}{2} \sum_{\mu} \left\{
%   (1 - \gamma_{\mu}) U_{\mu}(x) \psi(x + \hat{\mu}) + (1 + \gamma_{\mu}) U_{\mu}(x - \hat{\mu})^{-1} \psi(x - \hat{\mu})
% \right\} \;.
D_{\text{naive}}[U] = m + \frac{1}{2} \sum_{\mu} \gamma_{\mu} \left( D_{+\mu} + D_{- \mu} \right) \;,
\end{equation}
with the discretized covariant forward and backward derivatives
\begin{align}
%D_{+ \mu}(x,y) &= \frac{1}{a} \left[ U_{\mu}(x) \delta_{x+\hat{\mu},y} - \delta_{x,y} \right] \;, \\
%D_{- \mu}(x,y) &= \frac{1}{a} \left[ \delta_{x,y} - U_{\mu}(x - \hat{\mu})^{\dagger} \delta_{x - \hat{\mu}, y} \right] \;, \\
%D_{+ \mu} &= \frac{1}{a} \left[ U_{\mu} T_{\mu} - \id \right] \;, \\
%D_{- \mu} &= \frac{1}{a} \left[ \id - U_{-\mu} T_{-\mu} \right] \;, \\
%D_{\pm \mu} &= \pm \frac{1}{a} \left[ U_{\pm \mu} T_{\pm \mu} - \id \right] \;, \\
D_{\pm \mu}(x,y) &= \pm \frac{1}{a} \left[ U_{\pm \mu}(x) \delta_{x \pm \hat{\mu}, y} - \delta_{x,y} \right] \;,
%D_{+ \mu} f(x) &= \frac{1}{a} \left[ U_{\mu}(x) f(x + \hat{\mu}) - f(x) \right] \;, \\
%D_{- \mu} f(x) &= \frac{1}{a} \left[ f(x) - U_{\mu}(x - \hat{\mu})^{\dagger} f(x - \hat{\mu}) \right] \;,
\end{align}
and the $\gamma_{\mu}$ are the Dirac matrices obeying the Euclidean Clifford algebra $\{\gamma_{\mu}, \gamma_{\nu}\} = 2 \delta_{\mu \nu}$.

Since, the gauge field lives on the links between lattice points, the parallel transporter in positive direction $\mu$, linking point $x$ with its neighbor $x + \hat{\mu}$, is related to the link in negative $\mu$ direction linking point $x + \hat{\mu}$ and $x$ by Hermitian transpose.
This can be formalized as: for $\mu=1,2,3,4$ a direction
\begin{equation}
U_{\pm \mu}(x) = U_{\mp \mu}(x \pm \hat{\mu})^{\dagger} \;.,
\end{equation}
where $\pm$ denotes the link in positive or negative $\mu$-direction respectively.

Such a Dirac operator introduces fermion doublers; unphysical poles in the fermion propagator $D_{\text{naive}}^{-1}$.
These correspond to particles that do not exist in the continuum theory and the doublers make the operator ill-conditioned in general, unsuitable for numeric studies.
The Dirac operator can be extended by any term proportional to the lattice spacing $a$; such a term will vanish when taking the continuum limit $a \to 0$.
Wilson himself proposed a solution to the fermion doubling problem~\cite{PhysRevD.25.2649} resulting in the Wilson-Dirac operator
\begin{equation} \label{eq:dirac:wilson:clover}
D_W[U] = D_{\text{naive}}[U] - \frac{a}{2} \sum_{\mu} D_{-\mu} D_{+\mu} \;.
\end{equation}
This Wilson-term removed the fermion doubler states by making their mass proportional to $1/a$.
Such states decouple in the continuum limit, but the term above breaks chiral symmetry and introduces $\bigO(a)$ discretization errors\footnote{Even more troubling, every discretization with the correct continuum limit, respecting locality, either comes with fermion doublers or breaks chiral symmetry~\cite{Nielsen:1980rz,Nielsen:1981xu}}.

An additional term proportional to $a$ was introduced later to cancel the $\bigO(a)$ discretization effects introduced by the Wilson-term systematically while still eliminating the doublers,
\begin{equation}
D_{WC}[U] = D_W[U] + c_\mathrm{sw}^{SU(3)} a \frac{i}{4} \sum_{\mu, \nu} \sigma_{\mu \nu} \hat{F}_{\mu \nu} \;.
\end{equation}
This is called the Wilson-Clover Dirac operator.
Here, $\sigma_{\mu \nu} = \frac{i}{2} \left[\gamma_{\mu}, \gamma_{\nu}\right]$ and the discretized $SU(3)$ field strength tensor $\hat{F}_{\mu \nu}$ is defined as
\begin{equation}
\begin{aligned} \label{eq:intro:clover}
\hat{F}_{\mu \nu}(x) &= \frac{1}{8} \left\{
    Q_{\mu \nu}(x) - Q_{\nu \mu}(x)
\right\} \;, \\
Q_{\mu \nu}(x)
&= U_{\mu}(x)
   U_{\nu}(x+\hat{\mu})
   U_{\mu}(x+\hat{\nu})^{-1}
   U_{\nu}(x)^{-1} \\
&+ U_{\nu}(x)
   U_{\mu}(x-\hat{\mu}+\hat{\nu})^{-1}
   U_{\nu}(x-\hat{\mu})^{-1}
   U_{\mu}(x-\hat{\mu}) \\
&+ U_{\mu}(x-\hat{\mu})^{-1}
   U_{\nu}(x-\hat{\mu}-\hat{\nu})^{-1}
   U_{\mu}(x-\hat{\mu}-\hat{\nu})
   U_{\nu}(x-\hat{\nu}) \\
&+ U_{\nu}(x-\hat{\nu})^{-1}
   U_{\mu}(x-\hat{\nu})
   U_{\nu}(x+\hat{\mu}-\hat{\nu})
   U_{\mu}(x)^{-1} \;.
\end{aligned}
\end{equation}
The term proportional to $c_\mathrm{sw}^{SU(3)}$ is called the clover- or SW-term and was proposed by Sheikholeslami and Wohlert~\cite{sw1985} as addition to the action.
It is block-diagonal and gives $\bigO(a)$ improvement à la Symansik~\cite{symanzik1982,SYMANZIK1983} by tuning the coefficient such that $\bigO(a)$ discretization effects vanish.
Furthermore, the clover-term has a positive impact on the numerical condition.

There are many alternative discretizations to the above.
All deal differently with chiral symmetry on the lattice and the doubling problem.
Staggered fermions~\cite{PhysRevD.11.395} reduce the \num{16} doublers to \num{4} \emph{tastes} and retain a remnant \ggrp{U}{1} chiral symmetry.
Domain-wall fermions~\cite{Kaplan:1992bt,Shamir:1993zy} suppress doublers through the introduced fifth dimension of extent $L_s$ and exact chiral symmetry is obtained in the limit $L_s \to \infty$\footnote{Typically $L_s$ is chosen to be around \numrange{12}{48}.}.
Twisted-mass fermions~\cite{Frezzotti:2000nk} remove doublers with the Wilson term and chiral symmetry breaking effects are improved by the mass twisting.
Overlap fermions~\cite{Neuberger:1997fp,Neuberger:1998wv} entirely remove doublers and retain exact chiral symmetry.
This is in agreement with the Nielsen-Ninomiya no-go theorem~\cite{Nielsen:1980rz,Nielsen:1981xu}, because their Dirac operator is non-local making it computationally very expensive.

Summarizing, the QCD Wilson-Clover Dirac operator -- which for simplicity we will call $D$ in the remainder of this document -- is a nearest-neighbor stencil operator acting on a linear space of quark (or spinor) fields defined on a regular 4D cubic lattice of $V$ sites, typically in the range $10^6-10^8$.
Applied to a spinor field $\psi(x)$ the Dirac operator can then be written as (the lattice spacing is set to $a = 1$)
\begin{equation}
\begin{aligned} \label{eq:Dw}
(D &\psi)(x) = (4 + m) \psi(x) \\
-&\frac{1}{2} \sum_{\mu=0}^3 \Big\{
  U_{\mu}(x) (1-\gamma_{\mu}) \psi(x + \hat{\mu})
+ U_{\mu}(x-\hat{\mu})^{-1} (1+\gamma_{\mu}) \psi(x-\hat{\mu})
\Big\} \\
+&c_\mathrm{sw}^{SU(3)} \frac{i}{4} \sum_{\mu,\nu=0}^3 \sigma_{\mu \nu} F_{\mu \nu}(x) \psi(x),
\end{aligned}
\end{equation}
where the gauge field $U_{\mu}(x)$ is the $SU(3)$-valued link between lattice points $x$ and $x + \hat{\mu}$.

Notable about this choice of discretization are some properties which we briefly review here, because they will become important later.
The operator has the property of pseudo-Hermiticity with respect to $\gamma^5 = \gamma_0 \gamma_1 \gamma_2 \gamma_3$, $D^{\dagger} = \gamma^{5} D \gamma^{5}$, eigenvalues are either real or come in complex conjugate pairs and its determinant is real.

\subsection{Dirac equation}

Due to the nearest-neighbor coupling, a good parallelization scheme is achieved by domain decomposition, where a contiguous local volume of size $V_\mathrm{L}=L_0L_1L_2L_3$, which divides the global problem size, is associated with a single computing unit.
With accessible problem sizes, the Dirac operator is poorly conditioned and sophisticated preconditioning algorithms are needed to speed up the convergence of the iterative solvers used.

\tldr{Definition of Dirac eq on lattice}
The system of linear equations of interest is the Dirac equation
\begin{equation} \label{eq:dirac:equation}
  D \psi = \eta \;,
\end{equation}
where different choices for the right-hand side $\eta$ result in different propagator expressions.
Solves of this equation appear repeatedly throughout the field and will be a central computational motive in throughout this thesis.

\subsection{Lattice expectation values}

Taking our previous considerations into account, we find for Euclidean expectation values
\begin{equation} \label{eq:eclidean:exp:value}
\langle \observable{O}[\psi, \bar{\psi}, U] \rangle =
\frac{1}{Z}
\int \DD U
\det(D) e^{- S_G[U]}
\observable{O}[U] \;.
\end{equation}
The integral measure $\DD U$ is the Haar measure~\cite{haar1933} of the group manifold.
The fermion field dependencies of the observable in the integrand $\observable{O}[U]$ are Wick-contracted and replaced by traces over fermion propagators $D^{-1}$ from lattice points $x$ to $y$.
The observable now depends on solutions to the Dirac equation $D[U] \phi = \eta$, rather than the numerically inaccessible Grassmann-valued fermion fields $\psi(x)$ and $\bar{\psi}(x)$\footnote{This is a pattern occurring repeatedly in the field. As soon as fermions are introduced, expressions become non-local, like the determinant when integrating out the fermion fields or the appearance of $D^{-1}$ when Wick-contracting. This is the price to pay for a proper numerical treatment of Pauli's exclusion principle in Lagrangian dynamics. In Hamiltonian dynamics the price is an exponential growth of Hilbert space dimension where the boomerang might hit even harder.}.

The integral \cref{eq:eclidean:exp:value} is now amenable for numerical evaluation using Monte Carlo simulations.
We can define a probability density function as
\begin{equation}
P(U) = \frac{1}{Z} \det(D) e^{- S_G[U]} \;,
\end{equation}
importance sample a set of $\Nconf$ gauge fields according to it $\{U_i\}_{i=1}^{\Nconf}$ and estimate the expectation value of the observable as a mean over the samples
\begin{equation}
\langle \observable{O}[\psi, \bar{\psi}, U] \rangle \approx \frac{1}{\Nconf} \sum_{i=1}^{\Nconf} \observable{O}[U_i] \;.
\end{equation}

\subsection{C\texorpdfstring{$^{\star}$}{*} boundary conditions}

\tldr{intro cstar}
Gauss's law prohibits dynamical simulations of charged particles in a finite box with periodic boundary conditions.
A local and gauge-invariant formulation of QED theory on the lattice can be provided by using C$^{\star}$ boundary conditions along spatial lattice extents~\cite{cstar:Wiese1992,cstar:Polley1993,cstar:Kronfeld1991,cstar:Kronfeld1993}.
This theory is usually referred to as \QED{C}~\cite{Lucini:2015}.
It has some useful advantages like no zero-modes of the gauge field or small finite-volume effects due to its locality compared to other (non-local) QED formulations on the lattice.
Disadvantages are its partial breaking of flavor conservation and continuum and infinite volume limits do not commute, although mixing effects are exponentially suppressed.

\tldr{other QED lattice formulations and their pros/cons}
\QED{L}~\cite{BMW:2014pzb,10.1143/PTP.120.413} is local and removes the spatial zero modes of the photon~\cite{Lucini:2015}, but introduces potentially problematic non-localities in space, limits do not commute~\cite{Patella:2017fgk} and has unphysical finite-volume effects.
Furthermore, the fully local and consistent formulation \QED{m}~\cite{PhysRevLett.117.072002} introduces a photon of non-zero mass $\gamma_m$, whose limit $\gamma_m \to 0$ does not commute with the other limits.
\QED{TL}~\cite{Duncan:1996xy} is non-local in time and limit do not commute.
\QED{SF}~\cite{Gockeler:1989wj} restricts the zero mode of the photon field which allows charged states to propagate, but is non-local in time.
Infinite volume QED, QED$_{\infty}$~\cite{Asmussen:2016lse,Blum:2017cer,RBC_2018,Feng:2018qpx} subtracts \QED{L} long-range, power-law finite-volume effects proportional to $1/L$, but introduces non-localities.

%~\cite{10.1143/PTP.120.413,PhysRevLett.117.072002,Blum:2017cer,Feng:2018qpx}.

\tldr{Def of Cstar boundaries}
In this thesis, the prescription used is \QED{C} by employing spatial \Cstar boundary conditions to fields.
Fields are only periodic up to charge conjugation, that is
\begin{align} \label{eq:cstar:bcs}
  \begin{split}
    \psi(x + L_k \hat{k})       &= \psi^{\mathcal{C}}(x)       = C^{-1}\bar{\psi}^{\text{T}}(x), \\
    \bar{\psi}(x + L_k \hat{k}) &= \bar{\psi}^{\mathcal{C}}(x) = -\psi^{\text{T}}(x)C, \\
    A_{\mu}(x + L_k \hat{k})    &= A_{\mu}^{\mathcal{C}}(x)    = - A_{\mu}(x), \\
    U_{\mu}(x + L_k \hat{k})    &= U_{\mu}^{\mathcal{C}}(x)    = U_{\mu}^{*}(x),
  \end{split}
\end{align}
where $\psi$ is a fermion field, $A$ the $U(1)$-valued photon field and $U$ the $SU(3)$-valued gluon field and the subscript $\mathcal{C}$ stands for the charge conjugated field. The vector $\hat{k}$ is the unit vector in spatial $k$-direction with $k=1,2,3$, $\mu=0,1,2,3$ denotes the space-time direction and $L_k$ is the spatial lattice extent in direction $k$. The star symbol $U^{*}$ denotes element-wise complex conjugation. Finally $C$ is the invertible charge conjugation matrix obeying
\begin{equation}
  C^{-1} \gamma_{\mu} C = - \gamma_{\mu}^{\text{T}}
  \quad
  \text{and}
  \quad
  \det(C) = 1
\end{equation}
with the Euclidean $\gamma$-matrices. We would expect that charge conjugating the fields twice gives us back the original field. Thus shifting by twice the spatial lattice extent, the fermion field transforms as
\begin{align}
  \psi(x + 2 L_k \hat{k}) &= C^{-1}\bar{\psi^{\text{T}}}(x + L_k \hat{k}) = - C^{-1} C^{\text{T}} \psi(x), \\
  \bar{\psi}(x + 2 L_k \hat{k}) &= -\psi^{\text{T}}(x + L_k \hat{k})C = - \bar{\psi}(x) (C^{-1})^{\text{T}} C
\end{align}
If we chose the matrix $C$ to be skew-symmetric $C^{\text{T}} = -C$, the RHS equals $\psi(x)$ and $\bar{\psi}(x)$ respectively and the fermion field is periodic in space in twice the lattice extent $L_k$. Such a matrix $C$ always exists in four dimensions.

\tldr{physical, mirror and extended lattice}
Since $\psi$ and $\bar{\psi}$ are independent degrees of freedom, we see from \cref{eq:cstar:bcs} that a fermion field can be seen as a single degree of freedom on a larger lattice extended in all four spacetime directions by a factor of two.
The fermion field is then dynamic on that whole \emph{extended lattice} $\lat{ext}$ of extents $L_0$ and $2 L_{k}$, as compared to the gauge field which is only dynamic on the \emph{physical lattice} $\lat{phys}$ of extents $L_{\mu}$.
Finally, the part of the extended lattice where fields are charge conjugated will be called the \emph{mirror lattice} $\lat{mirr}$ such that $\lat{ext} = \lat{phys} \cup \lat{mirr}$.

\tldr{Extended lattice is requirement not implementation trick}
Therefore, the photon and the gluon field on the mirror lattice are completely determined by charge conjugating their values on the physical lattice. Their integration measure in the path integral \cref{eq:eclidean:exp:value} is given by
\begin{align}
\DD U \big\rvert_{\lat{phys}} &= \prod_{\mu=0}^{3} \prod_{x \in \lat{phys}} \dd U_{\mu}(x), \\
\DD A \big\rvert_{\lat{phys}} &= \prod_{\mu=0}^{3} \prod_{x \in \lat{phys}} \dd A_{\mu}(x).
\end{align}
On the other hand, $\psi$ and $\bar{\psi}$ are independent Grassmann variables on the physical lattice, whereas on the extended lattice $\bar{\psi}$ is completely determined by $\psi$. In the path integral the integration measure of the Grassmann variables therefore turns into
\begin{equation}
\DD \psi       \big\rvert_{\lat{phys}} \;
\DD \bar{\psi} \big\rvert_{\lat{phys}}
= \prod_{x \in \lat{phys}} \dd \psi(x) \dd \bar{\psi}(x)
= \prod_{x \in \lat{ext}}  \dd \psi(x)
= \DD \psi \big\rvert_{\lat{ext}} \;.
\end{equation}
The fermion field is dynamic on the whole extended lattice, where the physical and mirror lattices can also be seen as a further internal degree of freedom or an additional index $i \in \{ \text{physical}, \text{mirror} \}$ of the fermion field. 
%For that reason the fermion field defined on the extended lattice is sometimes called doublet.

\tldr{extended lattice is not 8x, but 2x larger}
One might be tempted to say that the extended lattice is \num{8} times larger than the physical one, making simulations with \Cstar boundary conditions significantly more expensive.
The data points on the mirror lattice though are redundant, because no matter in which spatial direction we leave the physical lattice in \cref{eq:cstar:bcs}, we always end up in the same mirror lattice.
By cleverly defining shifted boundary conditions on the extended lattice, one can thus get away with an extended lattice that is just twice as large as the physical one, irrespective on the number of spatial \Cstar directions.
This is called the orbifold construction~\cite{openqxd}.
Saying that \Cstar simulations are twice as expensive as periodic ones would still be ignorant to the many peculiarities of gauge field generation and observable evaluation.
Such a comparison is highly non-trivial as has to date not been fully carried out\footnote{One has to consider the missing experience in tuning these types of simulations, partial lack of theoretic foundation, differences in observable evaluation just to name a few. Also cost savings of incorporating QED dynamically, rather than perturbatively is not trivial to estimate. We will not even try an attempt.}.

\subsection{QCD+QED}

\tldr{QED in Dop}
For QCD+QED simulations, the Dirac operator \cref{eq:Dw} has to be modified to include electromagnetic interactions.
In addition to the \ggrp{SU}{3}-valued QCD gauge field $U_\mu(x)$, we have the \ggrp{U}{1}-valued QED gauge field $A_\mu(x)$, which when combined results in a \ggrp{U}{3}-valued field $e^{i q A_\mu(x)} U_\mu(x)$ with $q$ the charge of a quark. These links are produced by multiplying the \ggrp{U}{1} phase to the \ggrp{SU}{3} matrices.

\tldr{QED SW term}
In addition, we add another SW-term with separate coefficient,
\begin{equation} \label{eq:Dw2}
D \rightarrow D + q c_\mathrm{sw}^{U(1)} \frac{i}{4} \sum_{\mu,\nu=0}^3 \sigma_{\mu \nu} \hat{A}_{\mu \nu}\,,
\end{equation}
where $q$ is the charge and the $U(1)$ field strength tensor $\hat{A}_{\mu \nu}(x)$ is defined as
\begin{align*}
\hat{A}_{\mu \nu}(x) &= \frac{i}{4 q_{el}} \text{Im} \left\{
      z_{\mu \nu}(x)
    + z_{\mu \nu}(x-\hat{\mu})
    \right. \\
    &\phantom{=\frac{i}{4 q_{\text{el}}} \text{Im} \left\{ \right.} \left. 
    + z_{\mu \nu}(x-\hat{\nu})
    + z_{\mu \nu}(x-\hat{\mu}-\hat{\nu})
\right\}, \\
z_{\mu \nu}(x) &= e^{i\left\{
      A_{\mu}(x)
    + A_{\nu}(x+\hat{\mu})
    - A_{\mu}(x+\hat{\nu})
    - A_{\nu}(x)
\right\}},
\end{align*}
similar to the \ggrp{SU}{3} field strength with $q_{el} = \frac{1}{6}$ the elementary charge.

\tldr{Full QCD+QED Wilson Clover Dop}
Therefore the full QCD+QED Wilson-Clover Dirac operator applied to a spinor field $\psi$ is
\begin{equation}
\begin{aligned} \label{eq:Dw:QCD+QED}
D_\mathrm{w} \psi(x) = \left( 4 + m \right) &\psi(x) + \frac{1}{2} \sum_{\mu=0}^3
\Big\{
  \begin{multlined}[t]
    U_{\mu}(x)e^{i \hat{q} A_{\mu}(x)} (1-\gamma_{\mu}) \psi(x + \hat{\mu}) \\
   +U_{\mu}(x-\hat{\mu})^{-1}e^{-i \hat{q} A_{\mu}(x) } (1+\gamma_{\mu}) \psi(x-\hat{\mu})
\Big\} \end{multlined} \\
+&c_\mathrm{sw}^{SU(3)} \frac{i}{4} \sum_{\mu,\nu=0}^3 \sigma_{\mu \nu} \hat{F}_{\mu \nu}(x) \psi(x)
+q c_\mathrm{sw}^{U(1)} \frac{i}{4} \sum_{\mu,\nu=0}^3 \sigma_{\mu \nu} \hat{A}_{\mu \nu}(x) \psi(x),
\end{aligned}
\end{equation}
where the integer $\hat{q} = \frac{q}{q_{el}}$ the charge and $m$ is the bare quark mass.
Here, $x \in \lat{ext}$, thus the Dirac operator is acting on a spinor field defined on the extended lattice.

\section{Thesis objectives and scope}
\label{sec:intro:objectives}

This thesis aims to focus on the following objectives:
\begin{itemize}
   \item Design and efficient implementation of an offload of expensive solves of the Dirac equation to the GPU. This includes dealing with \Cstar boundary conditions and QCD+QED Dirac operators as of \cref{eq:Dw:QCD+QED}. It should happen in a modular and versatile fashion for convenient usage for application-level developers and lay foundation for future developments.
   \item Performance assessment and benchmark of all implemented features on modern HPC infrastructure including GPUs.
   \item Establishment of a CI/CD automated testing pipeline running on target hardware and production problems.
   \item Algorithmic development of variance reduction methods based on the low-mode deflation technique targeted at observables sensitive to low-mode noise suffering from the signal-to-noise ratio problem.
   \item Theoretic understanding of spectral properties of Dirac operators and their coarse-grid variants.
\end{itemize}
The above objectives all target efficient evaluation of observables from a machine-centric but also from an algorithmic perspective.

The scope of these developments concerns offloading inversions to the GPU and does not extend generally to the generation of gauge configurations although design decisions were made with that in mind.
A direct proposal for continuation of the software developments in that direction is stated in \cref{ch:p1:memory}, it is not the only possibility to continue.
Benchmarks are restricted to systems with NVIDIA GPUs, because of availability of compute time, although data for AMD GPUs has been investigated too, taken on the Finnish machine \emph{LUMI}.
Some of this can be found in ref.~\citeOwn{lattice24:roman} and currently unpublished in ref.~\citeOwn{pasc24}, where also data from the former \emph{Daint hybrid} machine at CSCS can be found.
This data is from an old version of the code and was left out of this document.

The first part of this document is dedicated to the advancements in software development by interfacing \openqxd with \quda.
The foundation of this work was my master's thesis~\citeOwn{lattice21:master_thesis,pasc23} and it is partially based on refs.~\citeOwn{pasc24,lattice24:roman}.

The second part investigates variance reduction using low-mode deflation techniques and is based on refs.~\citeOwn{mglma,lattice23:lma}.
Some of the plots and illustrations presented in this part were taken from ref.~\citeOwn{mglma}.
If so it is indicated in the captions.

A path not further pursued during my PhD was the evaluation of the leading order HVP of Muon $g-2$ using \Cstar boundary conditions and QCD+QED gauge ensembles.
Publications related to this can be found in refs.~\citeOwn{lattice22:roman,lattice22:paola,lattice22:alessandro,lattice22:jens,qcd_qed_paola}.

\section{Motivation}

Since modern HPC infrastructure tend towards massively parallel GPU systems or heterogeneous machines, research in computational sciences has the responsibility to utilize these new compute resources carefully.
If that does not happen a stagnation of the field is inevitable.
Efficient usage is only possible if current lattice gauge theory simulation applications are effectively running on GPUs.
As available pure-CPU compute resources are decreasing yearly, it is expected for them to vanish entirely in the not so far future.
Enabling \openqxd to run on GPUs thus opens important doors for the \RCstar collaboration and lays crucial groundwork for future developments.
It is evident that HPC hardware continues to change.
As soon as the GPU is harvested in terms of efficient usage, the HPC community may continue with practicing heterogeneous computing -- it already does.
Therefore we explored into this area as well.

In current world politics, energy consumption is more important than ever.
This obligates us as researchers to treat the resources we are allowed to use with extra care, specially in the field of computational lattice field theory -- one of the biggest consumers of HPC.

As important as software developments and efficient utilization of available resources are, the field will not advance without algorithmic developments.
Specially methods to systematically and iteratively improve estimations of observables by reducing their variance are pivotal for numeric high precision determination of physical quantities.
As traditional estimators either break down or become prohibitively expensive as we approach large, physical lattices, we need efficient variance reduction methods, that use the available resources more efficiently.

Summarizing the motivation, all aspects of this thesis crucially aim to maximize physics output, while minimizing cost in terms of computational resources and energy.

\section{Summary of Contributions}

The main contributions of this thesis are as follows:

\begin{itemize}
   \item \textbf{GPU integration into \openqxd}: Development of a feature to offload inversions to the solver suite of \quda into the core components of \openqxd targeted at observable evaluation. Implementation of \Cstar boundary conditions and the QCD+QED Wilson-Clover Dirac operator in \quda.
   \item \textbf{Versatile interface API}: Development of useful features such as the dual process grid (\cref{sec:interface:openqxd:dual}) or the asynchronous solver (\cref{sec:interface:solver:async}) allowing heterogeneous computing. The API was designed for easy porting of existing programs in \openqxd that aim to leverage solvers on the GPU.
   \item \textbf{Performance evaluation}: Performance of all implemented features was carefully measured on state-of-the-art HPC infrastructure and real-world production problems and setups.
   \item \textbf{CI/CD testing pipeline}: Integration of a CI/CD pipeline into the Git development-repository of \openqxd~\cite{gitlab:openqxd-devel} running on local compute resources as well as on GH200 nodes at CSCS.
   \item \textbf{Field memory management proposal}: Design and proof-of-concept test of a memory management system to minimize CPU-GPU field transfers useful for a continuation of this work considering gauge field generation.
   \item \textbf{Development of a novel variance reduction method}: Development, implementation and assessment of a new method called \emph{multigrid low-mode averaging} for $\bigO(a)$-improved Wilson-type fermions, specially suitable for the evaluation of the leading order connected light-quark HVP of the Muon $g-2$.
   \item \textbf{Theoretic considerations of chirality}: Investigation and analysis of chirality on spectral properties of coarse Dirac operators and their condition.
\end{itemize}

\section{Structure of the thesis}

This document is split into two parts which are to a large extent independent of each other, but pulling on the same rope.
\Cref{part:gpu} deals with the advancements in software development, GPU porting and the interface to \quda.
Its heart piece provides a documentation of the design choices all the functionality that have entered the codebase and finishes with an assessment of performance.
\Cref{part:variance} of this thesis introduces the new variance reduction method which we call \emph{multigrid low-mode averaging} and investigates its application to an observable that is sensitive to it.
Results are compared to traditional estimators.
It finishes with theoretical considerations about chirality on coarse subspaces.
Both parts have separate outlines and introductions.
