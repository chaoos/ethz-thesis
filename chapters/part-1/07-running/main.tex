\chapter{Running}
\label{ch:p1:running}

This section describes how to run the checks testing the interface manually\footnote{For automatic testing, please refer to \cref{ch:p1:cicd}.} and how the interface exposes itself to the user of openQxD once the software has been succesfully compiled.

\section{The input file}
\label{sec:running:infile}

The contents of the input file may be written in the traditional openQCD INI-inspired style. If the input file contains a section called \code{QUDA}, settings from that section will be read in in the bootstrapping phase, i.e. when calling \code{quda\_init()}. An example may look as follows:
\begin{minted}[]{ini}
[QUDA]
verbosity QUDA_VERBOSE # general verbosity
\end{minted}
This defaults to \code{QUDA\_VERBOSE} and may be overwritten by verbosities set in solver sections. Possible values include (in order or increasing verbosity) \code{QUDA\_SILENT}, \code{QUDA\_SUMMARIZE}, \code{QUDA\_VERBOSE} and \code{QUDA\_DEBUG\_VERBOSE}.

\section{Parametrizing a simple solver}
\label{sec:running:solver}

It remains to explain how the QUDA interface code expects the input file to look like. In our design choices, we made this as flexible as possible and compatible to legacy behavior of openQxD. Therefore a QUDA solver may be described in the same way as an openQCD solver. A minimal solver section may look as
\begin{minted}[]{ini}
[Solver 3]
solver          QUDA
maxiter         2048
gcrNkrylov      10
tol             1e-4
reliable_delta  1e-5
inv_type        QUDA_BICGSTAB_INVERTER
verbosity       QUDA_VERBOSE
solution_type   QUDA_MAT_SOLUTION
solve_type      QUDA_DIRECT_SOLVE
\end{minted}
The section name in the above example is \code{Solver 3}, thus the solver identifier is 3. This is the argument \code{id} that has to be provided when calling the solver with \code{openQCD\_qudaInvert}\worktodo{see where?}. For the section to be a valid QUDA solver section we need to specify the key-value pair \code{solver = QUDA}, else the setup function will throw an error message if fed with \code{id=3}. All the remaining keys parameterize the various members of the \code{QudaInvertParam} struct and we refer the reader to its documentation \cite{QUDApaper,github:quda}. We note that every name in \code{<quda>/include/quda\_enum.h} can be used as value for the key-value pairs in the input file for readability, so that the user doesn't have to write \code{2} for \code{QUDA\_GCR\_INVERTER} for instance. The example section above is minimal, i.e. it only sets required parameters; it solves the Wilson-Clover Dirac equation using a BiCGSTAB solver without any preconditioning to a relative residual of $10^{-4}$. We may have as many sections as the one above as desired enumerated by the unique identifier \code{id}.

\section{Parametrizing a multigrid solver}
\label{sec:running:multgrid}

This section describes a more involved example of a section that parametrizes a multigrid solver. It may look like
\begin{minted}[]{ini}
[Solver 4]
solver                  QUDA
maxiter                 256
gcrNkrylov              10
tol                     1e-12
reliable_delta          1e-5
inv_type                QUDA_GCR_INVERTER
verbosity               QUDA_VERBOSE
inv_type_precondition   QUDA_MG_INVERTER # use multigrid preconditioning
solution_type           QUDA_MAT_SOLUTION
solve_type              QUDA_DIRECT_SOLVE
mass_normalization      QUDA_MASS_NORMALIZATION
cuda_prec_sloppy        QUDA_SINGLE_PRECISION
cuda_prec_precondition  QUDA_HALF_PRECISION
\end{minted}
Different from the simple example in the previous section is that we specify a preconditioner via \code{inv\_type\_precondition}. Multigrid is special in the sense that it requires many more parameters. Namely the number of levels and a section of parameters for each level. This is specifed by a second section in the input file that may look as
\begin{minted}[]{ini}
[Solver 4 Multigrid]
n_level                 2
generate_all_levels     QUDA_BOOLEAN_TRUE
run_verify              QUDA_BOOLEAN_FALSE
compute_null_vector     QUDA_COMPUTE_NULL_VECTOR_YES
\end{minted}
This addtional section describes the multigrid preconditioner to consist of 2 levels. The name of the section is fixed to be \code{[Solver <N> Multigrid]} where \code{<N>} has to be substituted with the identifier of the solver section that specifies \code{inv\_type\_precondition=QUDA\_MG\_INVERTER}. The two levels are specified by two separate sections as
\begin{minted}[]{ini}
[Solver 4 Multigrid Level 0]
geo_block_size              4 4 4 4
n_vec                       24
n_vec_batch                 2
#verbosity                   QUDA_VERBOSE
spin_block_size             2
precision_null              QUDA_HALF_PRECISION
smoother                    QUDA_CA_GCR_INVERTER
smoother_tol                0.25
smoother_solve_type         QUDA_DIRECT_PC_SOLVE
nu_pre                      0
nu_post                     8
omega                       0.8
cycle_type                  QUDA_MG_CYCLE_RECURSIVE
coarse_solver               QUDA_GCR_INVERTER
coarse_solver_tol           0.25
coarse_solver_maxiter       50
coarse_grid_solution_type   QUDA_MATPC_SOLUTION
location                    QUDA_CUDA_FIELD_LOCATION

[Solver 4 Multigrid Level 1]
precision_null              QUDA_HALF_PRECISION
coarse_solver               QUDA_CA_GCR_INVERTER
smoother                    QUDA_CA_GCR_INVERTER
smoother_tol                0.25
smoother_solve_type         QUDA_DIRECT_PC_SOLVE
spin_block_size             1
coarse_solver_tol           0.25
coarse_solver_maxiter       50
coarse_grid_solution_type   QUDA_MATPC_SOLUTION
cycle_type                  QUDA_MG_CYCLE_RECURSIVE
nu_pre                      0
nu_post                     8
omega                       0.8
location                    QUDA_CUDA_FIELD_LOCATION
\end{minted}
Again, the section names are fixed to be \code{[Solver <N> Multigrid Level <M>]}, where \code{<N>} is the solver section identifier as above and \code{<M>} indicates the level, starting from zero being the finest coarse level. The parameters we can specify in the sections for the levels all those in the \code{QudaMultigridParam} struct that hold arrays of length \code{QUDA\_MAX\_MG\_LEVEL}, whereas all the other properties can be specified in the general multigrid section named \code{[Solver <N> Multigrid]}. Some properties have no effect on the finest coarse grid, some have no effect on the coarsest level. We refer the reader to the QUDA documentation for further details.

\worktodo{command to run?}
