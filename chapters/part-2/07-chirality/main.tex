\chapter{Spectral implications of chirality}
\label{ch:p2:chirality}

\readit{1}

% \worktodo{
% * Chirality
%     * Chirality plot
%     * Spectrum study plot
%     * Coarse Dop condition plot -> done in numerics->cost
%     * Eigen and Singular value decomps
%     * G5 Hermiticity
%     * Impl. chirality preservation -> done in lc->spin dof
%     * Spectral study
%     * All proofs I have so far
% }


%\worktodo{$[P,\chi^{5}]=0$, thus $P$ leaves the subspaces of fermion fields with definite chirality invariant.}

\newcommand{\csym}{\text{ID}} % Symbol for cases
\newcommand{\csyms}{\text{IDs}} % Symbols for cases
\newcommand{\boxcheck}{
    \makebox[0pt][l]{$\square$}\raisebox{.15ex}{\hspace{0.1em}$\checkmark$}
}
\newcommand{\convhull}[1]{\text{conv}\{ #1 \}}
\newcommand{\convshull}[1]{ \convhull{ \sigma( #1 ) } }

%\newcommand{\case}[1]{$\csym = #1$}
\newcommand{\caseX}[1]{$\csym$ (\hyperlink{id:#1}{#1})}
\newcommand{\pcaseX}[1]{$\csym$ (\protect\hyperlink{id:#1}{#1})}
\newcommand{\caseXtwo}[2]{$\csyms$ (\hyperlink{id:#1}{#1}) and (\hyperlink{id:#2}{#2})}
\newcommand{\caseXthree}[3]{$\csyms$ (\hyperlink{id:#1}{#1}), (\hyperlink{id:#2}{#2}) and (\hyperlink{id:#3}{#3})}
\newcommand{\caseXfour}[4]{$\csyms$ (\hyperlink{id:#1}{#1}), (\hyperlink{id:#2}{#2}), (\hyperlink{id:#3}{#3}) and (\hyperlink{id:#4}{#4})}
\newcommand{\rowtarget}[2]{\hypertarget{id:#1}{#2}}

% We define the \emph{condition number} of an operator $A$ as
% \begin{equation}
% \kappa(A) = \frac{ \sigma_{max}(A) }{ \sigma_{min}(A) } \;,
% \end{equation}
% where $\sigma_{max}(A)$ and $\sigma_{min}(A)$ are the largest and smallest singular values of $A$.
% We set $\kappa(A) = \infty$ if $\sigma_{min}(A) = 0$.

The main equation we want to establish in this section is
\begin{equation}
\kappa(\coarse{D}) \leq \kappa(D) \;.
\end{equation}
The condition number of the coarse Dirac operator should be bounded from above by the fine-grid condition number.
To achieve this, we need to discuss the spin degrees of freedom that are carried over to the coarse grid as they have significant algorithmic and theoretical implications.
There are many ways how this can be achieved.
We have already discussed some of them in \cref{sec:lc:spin}.
We want to formalize these findings more and come up with a sufficient condition for the coarse Dirac operator to have well-behaved spectrum.
To simplify things, we choose the Weyl basis for the $\gamma$-matrices (see \cref{eq:gamma:weyl:basis}), because it makes chirality manifest.
This choice makes $\gamma^5$ diagonal as
\begin{equation}
\gamma^5 =
\gamma_0 \gamma_1 \gamma_2 \gamma_3 =
\begin{pmatrix}
1 & 0 & 0 & 0 \\
0 & 1 & 0 & 0 \\
0 & 0 & -1 & 0 \\
0 & 0 & 0 & -1
\end{pmatrix} \;.
\end{equation}
It acts on positive chirality as identity and on negative chirality as minus identity.
By this the \num{4} spin degrees of freedom of a Dirac spinor are divided into chiral and non-chiral ones.
This makes the chiral degree of freedom manifest in $\gamma^5$ and we can write it as tensor product
\begin{equation} \label{eq:strict:separable}
\gamma^5 = \chi^5 \otimes \id_S \;,
\qquad
\chi^5 = 
\sigma_3 = 
\begin{pmatrix}
1 & 0 \\
0 & -1
\end{pmatrix} \;,
\qquad
\id_S = 
\begin{pmatrix}
1 & 0 \\
0 & 1
\end{pmatrix} \;,
\end{equation}
where $\chi^5$ only acts on the chiral degrees of freedom and the identity $\id_S$ on the remaining spins.
Such a decomposition is also possible in the Dirac- or Majorana basis.
The chiral operator $\chi^5$ will then be a different Pauli matrix.

On the fine grid, $\gamma^5$ only acts non-trivially on the spins.
Regarding notation, we usually suppress the identities in the tensor product when acting on spinors and morally identify
\begin{equation}
\gamma^5 \equiv \id_{\idxspacetime} \otimes \gamma^5 \otimes \id_{\idxcolor} \;,
\end{equation}
where $\id_{\idxspacetime}$ and $\id_{\idxcolor}$ are identities acting on the spacetime and color degrees of freedom respectively.
In this chapter, we may leave the identities explicit for clarity and define the (separable) chiral operator acting on a spinor as
\begin{equation}
\Gamma^5 = \id_{\idxspacetime} \otimes \chi^5 \otimes \id_S \otimes \id_{\idxcolor} \;,
\end{equation}
using the decomposition into lattice, chiral, non-chiral and color spaces.

We will assume throughout this chapter that we have determined $\Nc$ lowest modes of $\Gamma^{5} K$
\begin{equation} \label{eq:fine:low:modes}
\Gamma^5 K \evec_i = \lambda_i \evec_i \;,
\end{equation}
where $K$ is a $\Gamma^{5}$-Hermitian operator; $K^{\dagger} = \Gamma^{5} K \Gamma^{5}$.
Furthermore we have constructed coarse multigrid subspaces as described in \cref{ch:p2:lc} with restrictor $R$, prolongator $T=R^{\dagger}$ and Hermitian projector $P = TR$ as in \cref{eq:R,eq:T,eq:P}.

\section{Methodology}

For an explanation of the algorithm for numerical determinations of spectral hulls, numerical ranges and the symbols used in this section, please refer to \cref{ch:appendix:nr:ch:estimator}.
For numerical range determinations~\cite{johnson1978numerical}, we ignored all symmetries and used $N_{\theta}=128$ uniformly distributed angles for all operators
\begin{equation}
\Theta = \left\{ \frac{2 \pi n}{N_{\theta}} \mid n = 0, \ldots, N_{\theta}-1 \right\} \,.
\end{equation}
We have plotted the shaded region $W_{\text{out}}(A) \setminus W_{\text{in}}(A)$ as approximation to $\partial W(A)$.
Note that in the plots, this region is so small that it appears visually as a line and is in most cases even smaller than the width of the line.

The eigenvalues and eigenvectors of the Hermitian systems
\begin{equation}
H_{\theta} = \frac{1}{2} \left(e^{i \theta} A + e^{-i \theta} A^{\dagger} \right)
\end{equation}
were determined using the Generalized Davidson method with locally optimal restarting from the library PRIMME~\cite{primme}.
The eigenvalues close to some arbitrary shift $z \in \mathbb{C}$ for the non-normal fine grid Dirac operators were determined using the implicitly restarted Arnoldi algorithm~\cite{doi:10.1137/S0895479899358595} implemented in \quda\footnote{The shifts had to be implemented manually.}, whereas for the non-normal coarse grid operators we used the Krylov-Schur method implemented in SLEPc~\cite{slepc}.

The condition numbers were determined by estimating the largest and smallest magnitude eigenvalues of the HPD operator $A^{\dagger} A$, whose square root equal the largest and smallest singular values of $A$.
For this again the Generalized Davidson method with locally optimal restarting from the library PRIMME~\cite{primme} was used.

% \worktodo{error estimates of NR and CH}
% \worktodo{how num range was determined~\cite{johnson1978numerical}, how conv spec hull was determined, eigensolvers: quda, primme, SLEPc}



\section{Coarse chirality}

Since $\Gamma^5$ is an operator acting on Dirac spinors
\begin{equation}
\Gamma^5 \colon \vslattice \longrightarrow \vslattice \;,
\end{equation}
we can coarsen it using the usual formula, \cref{eq:sd:coarse:op},
\begin{equation}
\coarse{\Gamma^5} = R \Gamma^5 T \colon \coarse{\vslattice} \longrightarrow \coarse{\vslattice} \;,
\end{equation}
where $R$ and $T$ are restrictor and prolongator defined in \cref{eq:R,eq:T}.
%Similarly as with the fine operator, we can write the coarse operator as tensor product of identities with a (possibly) non-trivial chiral operator \worktodo{sure its not $\id \otimes \rho^{5}(\coarse{x})$?}
Similarly as the fine chirality operator, the coarse one is diagonal in spacetime.
The remaining indices though are non-trivial
\begin{equation}
%\coarse{\Gamma^5} = \id_{\coarse{\idxspacetime}} \otimes \rho^5 \;,
\opxy{\coarse{\Gamma^5}}{\coarse{x}}{\coarse{y}} = \delta_{\coarse{x} \coarse{y}} \cdot \rho^5(\coarse{x}) \;,
\end{equation}
where $\rho^{5}(\coarse{x})$ acts only on coarse spins and coarse colors, potentially different for every coarse lattice point $\coarse{x}$.
%where $\id_{\coarse{\idxspacetime}}$ is the identity on the coarse spacetime and $\rho^{5}$ acts only on coarse spins and coarse colors.
%The identities themselves are not of interest, but the coarse chirality operator $\rho^5$ is.
As opposed to the fine grid, we loose the strict separability as of \cref{eq:strict:separable} of $\rho^5$ into spin (chiral and non-chiral indices) and color in general.

The spin degrees of freedom can be coarsened in many ways.
%If we coarsen all of them, $N_s = 1$, then the two operators $\rho^{5}$ and $\id_{\coarse{S}}$ are both $1 \times 1$ matrices, i.e. numbers and the coarse $\coarse{\Gamma^5}$ is proportional to the identity.
%If we coarsen all of them, $N_s = 1$, then $\rho^{5}$ only has coarse color indices.
The are certain coarsening choices that reinstantiate the strict separability of the fine operator and make $\rho^5(\coarse{x})$ independent of $\coarse{x}$.
%If we coarsen none of them by leaving all \num{4} spins explicit on the coarse subspace $N_s = 4$, then both operators will be $2 \times 2$ matrices equal to their fine grid versions, $\rho^{5} = \chi^{5}$ and $\id_{\coarse{S}} = \id_S$.
%Finally, if we coarsen using the chiral projectors \cref{eq:bootstrap:chiral} leaving the chiral indices explicit and coarsening the remaining spin indices $N_s=2$, we again find $\rho^{5} = \chi^{5}$ and $\id_{\coarse{S}} = 1$ as a number.
%Alternatively we could coarsen the chiral indices and leave the non-chiral ones explicit.
%This would analogously result in $\rho^{5} = 1$ and $\id_{\coarse{S}} = \id_S$.
\Cref{tab:spins} summarizes most of the sensible choices and some of their properties.
We explicitly distinguish between chiral and non-chiral indices.
We see that some preserve the trace on the fine-grid $\tr(\Gamma^{5})=0$, some preserve chirality $[P, \Gamma^{5}] = 0$ and some preserve fine-grid low modes $P \evec_i = \evec_i$, \cref{eq:fine:low:modes}.

Even though case \caseX{2wc} produces a Hermitian coarse operator $\coarse{K}$, it falls out immediately, since the coarse subspace must contain the low modes (last column).
This is a restriction of LMA and multigrid LMA that by design require to capture the low mode subspace.
Projecting parts away would be counter productive.
However, the remaining choices are all valid.
Cases \caseX{2sc} and \caseX{2sv} are physically equivalent, they only differ in basis.
When appearing in traces, they evaluate to the same result.
Case \caseX{1} preserves no spin degrees of freedom, case \caseX{2nc} preserved the non-chiral ones and cases \caseXfour{2sc}{2sv}{2wc}{4} preserve chiral degrees of freedom, \caseX{4} even both.

Preservation of the trace $\tr(\coarse{\Gamma^5})=0$ implies that both chiral degrees of freedom are treated symmetrically, i.e. both are explicit on coarse subspaces, none is preferred over the other.
We therefore define the term \df{weak chirality preservation} of coarse subspaces as subspaces whose projector obeys
\begin{equation}
[P, \Gamma^{5}] = 0 \;.
\end{equation}
If additionally $\tr(\coarse{\Gamma^5}) = 0$, we call it \df{strong chirality preservation}.
Only cases \caseXthree{2sc}{2sv}{4} strongly preserve chirality in that way.
We will find that only these cases lead to well condition coarse operators $\coarse{K}$.

%\worktodo{say that cases 2,3 are equivalent, both cases 5 are stupid because does not preserve modes, but create a Hermitian coarse Dirac operator, subspace containment: $\coarse{\vslattice}_{\text{\caseX{1}}} \subseteq \coarse{\vslattice}_{\text{\caseX{2sc}}}, \ldots$, case 2,3 preserve chiral dof, case 4 preserved non-chiral dofs, case 6 preserves both}

% \begin{table}
% \begin{tabular}{c|cccccc}
% \toprule
% \csym & $N_s$ & Spins & $\rho^{5}$ & $\tr(\coarse{\Gamma^5})$ & $[P, \Gamma^{5}]$ & $P\evec_i = \evec_i$ \\
% \midrule
% 1 & 1 & - & Hermitian & $\in \mathbb{R}$ & $\neq 0$ & \boxcheck \\
% \midrule
% 2 & 2 & $P_{\pm} = \frac{1}{2} \left( \id \pm \Gamma^{5} \right)$ & $\chi^{5} \otimes \id_{\coarse{\idxcolor}}$ & $0$ & $0$ & \boxcheck \\
% \midrule
% 3 & 2 & $P_i = \id, P_g = \Gamma^{5}$ & $B \left(\chi^{5} \otimes \id_{\coarse{\idxcolor}} \right) B^{-1}$ & $0$  & $0$ & \boxcheck \\
% % \midrule
% % 4 & 2 &
% % %\(\displaystyle
% % \makecell{
% %   $P_{e} = P_0 + P_2$, \\
% %   $P_{o} = P_1 + P_3$
% % }
% % %(P_{1})_{[\alpha \beta]} = \delta_{\alpha \beta} \left( \delta_{\alpha 1} + \delta_{\alpha 3} \right)
% % %\)
% % & \worktodo{?} & \textcolor{red}{?} & \textcolor{red}{?} & \boxcheck \\
% \midrule
% % 4 & 2 & $P_0, P_1$ & $+\id_{\coarse{S}} \otimes \id_{\coarse{\idxcolor}}$ & $+2 \Nc$ & $0$ & $\square$ \\
% % \midrule
% % 5 & 2 & $P_2, P_3$ & $-\id_{\coarse{S}} \otimes \id_{\coarse{\idxcolor}}$ & $-2 \Nc$ & $0$ & $\square$ \\
% % \midrule
% 4 & 2 & $P_{0,2}, P_{1,3}$ & Hermitian & $\in \mathbb{R}$ & $\neq 0$ & $\boxcheck$ \\
% \midrule
% 5 & 2 & $P_0, P_1$ or $P_2, P_3$ & $\pm \id_{\coarse{S}} \otimes \id_{\coarse{\idxcolor}}$ & $\pm 2 \Nc$ & $0$ & $\square$ \\
% \midrule
% 6 & 4 & $P_0, P_1, P_2, P_3$ & $\chi^{5} \otimes \id_{\coarse{S}} \otimes \id_{\coarse{\idxcolor}}$ & $0$ & $0$ & \boxcheck \\
% \bottomrule
% \end{tabular}
% \caption{\label{tab:spins}
% Projectors together with their number of coarse spin degrees of freedom $N_s$ and the form of coarse chiral and non-chiral operators. The table uses the definition of the spin projectors $(P_{\sigma})_{[\alpha \beta]} = \delta_{\alpha \beta} \delta_{\alpha \sigma}$ and $P_{\sigma,\rho} = \frac{1}{2}(P_{\sigma} + P_{\rho})$for $\sigma,\rho=0,1,2,3$. The matrix $B$ is some irrelevant change-of-basis matrix.
% }
% \end{table}

\begin{table}
\begin{tabular}{l|ccccc}
\toprule
$\csym$ & Spins & $\rho^{5}(\coarse{x})$ & $\tr(\coarse{\Gamma^5})$ & $[P, \Gamma^{5}]$ & $P\evec_i = \evec_i$ \\
\midrule
\rowtarget{1}{1} & - & Hermitian & $\in \mathbb{R}$ & $\neq 0$ & \boxcheck \\
\midrule
\rowtarget{2sc}{2sc} & $P_{\pm} = \frac{1}{2} \left( \id \pm \Gamma^{5} \right)$ & $\chi^{5} \otimes \id_{\coarse{\idxcolor}}$ & $0$ & $0$ & \boxcheck \\
\midrule
\rowtarget{2sv}{2sv} & $P_i = \id, P_g = \Gamma^{5}$ & $B_{\coarse{x}} \left(\chi^{5} \otimes \id_{\coarse{\idxcolor}} \right) B_{\coarse{x}}^{-1}$ & $0$  & $0$ & \boxcheck \\
\midrule
\rowtarget{2nc}{2nc} & $P_{0,2}, P_{1,3}$ & Hermitian & $\in \mathbb{R}$ & $\neq 0$ & $\boxcheck$ \\
\midrule
\rowtarget{2wc}{2wc} & $P_0, P_1$ or $P_2, P_3$ & $\pm \id_{\coarse{S}} \otimes \id_{\coarse{\idxcolor}}$ & $\pm 2 \Nc$ & $0$ & $\square$ \\
\midrule
\rowtarget{4}{4} & $P_0, P_1, P_2, P_3$ & $\chi^{5} \otimes \id_{\coarse{S}} \otimes \id_{\coarse{\idxcolor}}$ & $0$ & $0$ & \boxcheck \\
\bottomrule
\end{tabular}
\caption{\label{tab:spins}
Projectors and the form of coarse chiral and non-chiral operators.
The table uses the definition of the spin projectors $(P_{\sigma})_{[\alpha \beta]} = \delta_{\alpha \beta} \delta_{\alpha \sigma}$ and $P_{\sigma,\rho} = \frac{1}{2}(P_{\sigma} + P_{\rho})$for $\sigma,\rho=0,1,2,3$.
The matrix $B$ is some irrelevant block-diagonal change-of-basis matrix.
The $\csyms$ stand for Nxx, where N denotes the number of coarse spin degrees of freedom $\Ns$ and the letters xx stand for sc: strongly chiral, nc: non-chiral, wc: weakly chiral and sv: singular value decomposition.
}
\end{table}

%\worktodo{Go to cell \caseX{1}, \caseXtwo{1}{4}, \caseXthree{1}{2sc}{4} in the table, balbalb \cref{eq:gamma:weyl:basis,eq:nr:linear,eq:nr:dagger} test.}
We want to further investigate the properties of the coarse chirality operator $\coarse{\Gamma^{5}}$.
For this we look at its eigenvalues and eigenvectors,
\begin{equation}
\coarse{\Gamma^{5}} \coarse{\psi} = \lambda \coarse{\psi} \;.
\end{equation}
Plugging in the definitions of the coarse objects, we obtain
\begin{equation}
R \Gamma^{5} \psi = \lambda R \psi \;,
\end{equation}
where we used $P = TR$ and the fact that for every coarse spinor $\coarse{\psi}$ we have $P \psi = \psi$ for its prolonged spinor $\psi = T \coarse{\psi}$.
This is just an equation restricted to the subspace and we can prolong it to the fine grid by applying $T$ to both sides,
\begin{equation}
P \Gamma^{5} \psi = \lambda \psi \;.
\end{equation}
If chirality is weakly preserved, $[P, \Gamma^{5}] = 0$, we find the eigenvalue equation for the fine-grid chirality operator $\Gamma^{5} \psi = \lambda \psi$.
%We will now use a property which will be discussed later, $[P, \Gamma^{5}] = 0$, and find the eigenvalue equation for the fine-grid chirality operator $\Gamma^{5} \psi = \lambda \psi$.
Therefore, if $[P, \Gamma^{5}] = 0$ the spectrum of the coarse chirality operator is a subset of the fine one; explicitly either
\begin{equation} \label{eq:sigma}
\sigma(\coarse{\Gamma^{5}}) = \{+1\} \;,
\qquad
\sigma(\coarse{\Gamma^{5}}) = \{-1\} \;,
\qquad
\text{or}
\qquad
\sigma(\coarse{\Gamma^{5}}) = \{+1, -1\} \;.
\end{equation}
Clearly for the first two spectra the coarse chirality operator will be proportional to the identity and will not be traceless corresponding to both cases \caseX{2wc} in \cref{tab:spins}.
A trace-preserving operator appears if both eigenspaces of the fine chirality operator are coarsened symmetrically.

% We will continue with defining the \df{net-chirality} of a field by the number
% \begin{equation}
% \chi(\psi) = \frac{\psi^{\dagger} \Gamma^{5} \psi}{\psi^{\dagger} \psi} \in [-1, 1] \;.
% \end{equation}
% The set of values for all possible $\psi \in \vslattice$ is the \df{numerical range},
% \begin{equation}
% W(\Gamma^{5}) = \left\{ \frac{\psi^\dagger \Gamma^{5} \psi}{\psi^{^\dagger} \psi} \; \middle| \; \psi \in \vslattice \right\} = [-1, 1] \;.
% \end{equation}
% This implies that if $\chi(\psi) > 0$ the field is biased toward right-handed and if $<0$ towards left-handed chirality.

\begin{lemma}[Weak chirality preservation] \label{lemma:chirality:preservation:equiv}
Assuming that $P$ and $\Gamma^{5}$ are Hermitian, the following statements are all equivalent:
\begin{align}
[P, \Gamma^{5}] &= 0 \;, &\text{(weak chirality preservation)} \label{eq:lemma:chirality:preservation:equiv} \\
\sigma(\coarse{\Gamma^{5}}) &\subseteq \sigma(\Gamma^{5}) \;, \label{eq:lemma:sigma} \\
(\coarse{\Gamma^{5}})^{2} &= \id \;, &\text{(involution)} \label{eq:lemma:involution} \\
\coarse{\Gamma^{5}} (\coarse{\Gamma^{5}})^{\dagger} &= (\coarse{\Gamma^{5}})^{\dagger} \coarse{\Gamma^{5}} = \id \;, &\text{(unitary)} \label{eq:lemma:unitary} \\
\tr([P, \Gamma^{5}]^{2}) &= 0 \label{eq:lemma:trsq0} \;.
\end{align}
\end{lemma}
\begin{proof}
Assuming the first equality to be true, we have already shown that the spectrum has to be one of \cref{eq:sigma}.
Thus \cref{eq:lemma:chirality:preservation:equiv} implies \cref{eq:lemma:sigma}.

Looking at an operator with one of the three possible spectra, all of them square to an operator with spectrum $\{1\}$, which is the identity operator.
Thus \cref{eq:lemma:sigma} implies \cref{eq:lemma:involution}.

Coarse operators inherit Hermiticity from fine ones.
Thus a Hermitian involution is unitary; \cref{eq:lemma:involution} implies \cref{eq:lemma:unitary}.

Assuming we have a unitary, we calculate straight,
\begin{align}
\tr([P, \Gamma^{5}]^{2})
&= 2 \left[ \tr(P \Gamma^{5} P \Gamma^{5})  - \tr(P) \right] \\
&= 2 \left[ \tr(R \Gamma^{5} T R \Gamma^{5} T)  - \rank(P) \right] \\
&= 2 \left[ \tr( \coarse{\Gamma^{5}} \coarse{\Gamma^{5}} )  - \dim(\coarse{\vslattice}) \right] \\
&= 2 \left[ \tr( (\coarse{\Gamma^{5}})^{\dagger} \coarse{\Gamma^{5}} )  - \dim(\coarse{\vslattice}) \right] \\
&= 2 \left[ \tr( \id )  - \dim(\coarse{\vslattice}) \right] \\
&= 0 \;,
\end{align}
where in the last step we used the unitary property of $\coarse{\Gamma^{5}}$.
Thus \cref{eq:lemma:unitary} implies \cref{eq:lemma:trsq0}.

Finally, assuming \cref{eq:lemma:trsq0} and Hermiticity of $P$ and $\Gamma^{5}$, we find that
\begin{equation}
[P, \Gamma^{5}]^{\dagger} = - [P, \Gamma^{5}] \;,
\end{equation}
is skew-Hermitian.
Its eigenvalues are pure imaginary, $\sigma([P, \Gamma^{5}]) = \{ i \lambda_j\}_j$ for $\lambda_j \in \mathbb{R}$.
The trace over the operator squared is zero by assumption, thus
\begin{equation}
0 = \sum_{j} (i \lambda_j)^2 = - \sum_{j} \lambda_j^2 \;.
\end{equation}
This is only possible if $\lambda_j=0$ for all $j$.
Thus \cref{eq:lemma:trsq0} implies \cref{eq:lemma:chirality:preservation:equiv} concluding the proof.
\end{proof}

\section{Weak chirality preservation}

%Even though we know that the property of chirality preservation $[P, \Gamma^{5}]=0$ is not sufficient for the coarse operator to be well-conditioned, it is worth inspecting it further.
The property of weak chirality preservation $[P, \Gamma^{5}]=0$ is worth further inspection.

\begin{lemma}[Implications of weak chirality preservation] \label{lemma:chirality:preservation:implications}
Assuming weak chirality preservation and $K$ is a $\Gamma^{5}$-Hermitian operator on the fine-grid lattice, then it holds
\begin{align}
\coarse{\Gamma^{5}} \coarse{K} &= \coarse{\Gamma^{5} K} \;, \\
\coarse{K} \coarse{\Gamma^{5}} &= \coarse{K \Gamma^{5}} \;, \\
\coarse{K} \; \text{is} \; &\coarse{\Gamma^{5}} \text{-Hermitian.}
\end{align}
\end{lemma}

\begin{proof}
The first equation is readily shown as
\begin{align}
\coarse{\Gamma^{5}} \coarse{K} = R \Gamma^{5} T R K T = R \Gamma^{5} P K T = R P \Gamma^{5} K T = R \Gamma^{5} K T = \coarse{\Gamma^{5} K} \;.
\end{align}
The second equation is just the same again, and for the third equation we note that Hermiticity is preserved when coarsening and since $\Gamma^{5} K$ is Hermitian by assumption, so is $\coarse{\Gamma^{5} K}$ and $\coarse{\Gamma^{5}} \coarse{K}$, thus $\coarse{K}^{\dagger} = \Gamma^{5} \coarse{K} \Gamma^{5}$.
\end{proof}

These are properties we may wish to have on the coarse grid, but we note that for case \caseX{2wc} in \cref{tab:spins} this is satisfied, although $\coarse{K}$ might become ill-conditioned as seen later.

\subsection{Singular value decomposition}

On the fine-grid, the singular value decomposition (SVD) of a $\Gamma^{5}$-Hermitian operator $K$ is
\begin{equation} \label{eq:svd}
K = \sum_{i} \lvert \lambda_i \rvert (\Gamma^{5} \psi_i) \tilde{\psi}_i^{\dagger} \;,
\end{equation}
where the $\psi_i$ and $\lambda_i$ are eigenmodes and eigenvalues of the Hermitian operator $\Gamma^{5} K$ and $\tilde{\psi}_i = \text{sign}(\lambda_i) \psi_i$.
%The magnitudes of the eigenvalues $\lvert \lambda_i \rvert$ are the singular values of $K$.
The singular values $\sigma_i = \lvert \lambda_i \rvert$ are the magnitudes of eigenvalues $\lambda_i$ of $\Gamma^{5} K$.
%The SVD above is not unique, but the singular values are.

\begin{lemma}[Coarse singular value decomposition] \label{lemma:svd:coarse}
If chirality is weakly preserved, $[P, \Gamma^5]=0$, the singular value decomposition of $\coarse{K}$ is equivalent to \cref{eq:svd}, with every operator and vector hatted.
\end{lemma}

\begin{proof}
Weak chirality preservation implies $\coarse{\Gamma^{5}}$-Hermiticity of $\coarse{K}$ and $\coarse{\Gamma^{5}}$ is an involution.
As in the fine case, we look at the eigenvalue decomposition of the coarse Hermitian operator,
\begin{equation}
\coarse{\Gamma^{5} K} = \coarse{\Gamma^{5}} \coarse{K} = \sum_{i} \lambda_i \coarse{\psi}_i \coarse{\psi}_i^{\dagger} \;,
\end{equation}
with real eigenvalues $\lambda_i$ and eigenvectors $\psi_i$. Using that $\coarse{\Gamma^{5}}$ is an involution, we obtain
\begin{equation}
\coarse{K} = \sum_{i} \lambda_i \coarse{\Gamma^{5}} \psi_i \psi_i^{\dagger} \;,
\end{equation}
and with definitions $\sigma_i = \lvert \lambda_i \rvert$ and $\coarse{\tilde{\psi}}_i = \text{sign}(\lambda_i) \coarse{\psi}_i$ we get \cref{eq:svd} with hatted quantities.
\end{proof}

An implication of \cref{eq:svd} is that the condition numbers of the non-Hermitian and Hermitian operators are all equal
\begin{equation}
\kappa(K) = \kappa(\Gamma^{5} K) = \kappa(K \Gamma^{5}) \,,
\end{equation}
and according to \cref{lemma:svd:coarse} the same holds for their weak chirality preserving coarse variants.
Regarding coarse operator condition, it therefore does not matter if we coarsen the Dirac operator itself or its Hermitian version, $Q=\Gamma^{5} D$.

\subsection{Eigenvalues and determinant}

$\Gamma^{5}$-Hermiticity of $K$ also impacts the determinant and eigenvalues.
A careful analysis reveals that the determinant is real and eigenvalues are either real or come in complex conjugate pairs.

% \worktodo{Furthermore, we will assume the operator $K$ to be \df{accretive}, meaning that $\Re(\psi^{\dagger} K \psi) > 0$ for all normalized $\psi$.
% This also implies that $K + K^{\dagger}$ is HPD.
% For an accretive operator, the numerical range $W(K)$ lies in the closed complex right half-plane, making it positively stable.
% This is a sane assumption for the Wilson-Clover Lattice Dirac operator $D$ from physical or larger pion mass simulations.}

\begin{lemma}[Coarse eigenvalues and determinant] \label{lemma:evals:coarse}
%$\coarse{K}$ is accretive and positively stable if $K$ is.
If chirality is weakly preserved, eigenvalues of $\coarse{K}$ are either real or come in complex conjugate pairs and its determinant is real.
\end{lemma}

\begin{proof}
%Accretivity and positive stability follow from the fact that the numerical range of $\coarse{K}$ is contained in the numerical range of $K$, $W(\coarse{K}) \subseteq W(K)$.
On the coarse grid, we can directly infer the determinant as $\det(\coarse{\Gamma^{5}}) = \pm 1$, because of \cref{eq:sigma,lemma:chirality:preservation:equiv}.
Then for the determinant of $\coarse{K}$, we find
\begin{equation}
\det(\coarse{K})^{\star} =
\det(\coarse{K}^{\dagger}) =
\det(\coarse{\Gamma^{5}})^{2} \det(\coarse{K}) =
\det(\coarse{K}) \;,
\end{equation}
i.e. the determinant is real and eigenvalues are either real or come in complex conjugate pairs as on the fine grid.
\end{proof}

% \worktodo{
% \Cref{lemma:evals:coarse} also implies that $\coarse{K}$ is invertible (at least analytically), since its spectrum lies on the closed complex right half-plane which excludes exact zero modes.
% Nevertheless $\coarse{K}$ may be ill-conditioned, because its smallest magnitude eigenvalue -- even though non-zero -- can still lie arbitrarily close to the complex origin.
% }

Summarizing \cref{lemma:chirality:preservation:implications,lemma:svd:coarse,lemma:evals:coarse}, weak chirality preservation transfers many properties from the fine to the coarse grid.
To represent the Dirac operator on the coarse grids properly, it is clearly not a bad idea to preserve chirality.
From \cref{tab:spins}, we see that all cases except \caseX{1} lead to $[P, \Gamma^{5}]=0$.
On the other hand, case \caseX{2wc} inherits all the properties, but projects to only one of the two chiralities.
%The trace and the low modes are not preserved.

\section{Numerical range}

Before we analyze coarsenings and operators, we need some definitions.

\begin{enumerate}
\item{
The \df{numerical range} of an operator $A \colon \vslattice \rightarrow \vslattice$ is defined as
\begin{align} \label{eq:def:numerical:range}
W(A) \coloneqq \left\{ \psi^\dagger A \psi \; \middle| \; \psi \in \vslattice \; \text{and} \; \norm{\psi} = 1 \right\}.
\end{align}
}
\item{The \df{open right-half plane} is defined as $\mathbb{H} = \{ z \in \mathbb{C} \colon \Re(z) > 0 \}$.}
\item{An operator $A$ is called \df{accretive} if its numerical range is a subset of the open right-half plane; $W(A) \subset \mathbb{H}$.}
% \item{
% The \df{numerical radius} of an operator $A$ is
% \begin{equation}
% \nrad{A} =
% %\max_{z \in W(A)} \lvert z \rvert \;.
% \sup \{ \lvert z \rvert ,\; z \in W(A) \} \;.
% \end{equation}
% }
% \item{
% The \df{inner numerical radius} of an operator $A$ is
% \begin{equation}
% \inrad{A} =
% \inf \{ \lvert z \rvert ,\; z \in W(A) \} \;.
% \end{equation}
% }
\item{
Finally, the \df{convex hull} of a set of complex numbers $\mathcal{A} \subset \mathbb{C}$ is defined as the smallest convex set in $\mathbb{C}$ that contains $\mathcal{A}$, mathematically
\begin{equation}
\convhull{\mathcal{A}} = \left\{
    \sum_{j=1}^{n} \lambda_j z_j
    \; \middle| \;
    z_j \in \mathcal{A}, \;
    \lambda_j \geq 0, \;
    \sum_{j=1}^{n} \lambda_j = 1, \;
    n \in \mathbb{N}
\right\} \;.
\end{equation}
This is the union of all convex combinations of points in $\mathcal{A}$.
}
\end{enumerate}

% The \df{numerical range} of an operator $A \colon \vslattice \rightarrow \vslattice$ is defined as
% \begin{align} \label{eq:def:numerical:range}
% W(A) \coloneqq \left\{ \psi^\dagger A \psi \; \middle| \; \psi \in \vslattice \; \text{and} \; \norm{\psi} = 1 \right\}.
% \end{align}
% The \df{open right-half plane} is defined as $\mathbb{H} = \{ z \in \mathbb{C} \colon \Re(z) > 0 \}$.
% An operator $A$ is called \df{accretive} if its numerical range is a subset of the open right-half plane; $W(A) \subseteq \mathbb{H}$.
% The \df{numerical radius} of an operator $A$ is
% \begin{equation}
% \nrad{A} =
% %\max_{z \in W(A)} \lvert z \rvert \;.
% \sup \{ \lvert z \rvert ,\; z \in W(A) \} \;,
% \end{equation}
% and the \df{inner numerical radius} of an operator $A$ is
% \begin{equation}
% \inrad{A} =
% \inf \{ \lvert z \rvert ,\; z \in W(A) \} \;.
% \end{equation}
% Finally, the \df{convex hull} of a set of complex numbers $\mathcal{A} \subset \mathbb{C}$ is defined as the smallest convex set in $\mathbb{C}$ that contains $\mathcal{A}$, mathematically
% \begin{equation}
% \convhull{\mathcal{A}} = \left\{
%     \sum_{j=1}^{n} \lambda_j z_j
%     \; \middle| \;
%     z_j \in \mathcal{A}, \;
%     \lambda_j \geq 0, \;
%     \sum_{j=1}^{n} \lambda_j = 1, \;
%     n \in \mathbb{N}
% \right\} \;.
% \end{equation}
% This is the union of all convex combinations of points in $\mathcal{A}$.

\begin{theorem}[The numerical range] \label{thm:nr:properties}
This theorem collects some standard properties and results about the numerical range of an operator $A \colon \vslattice \rightarrow \vslattice$, that will become important in this chapter.
For proofs, see~\cite{gustafson1997numerical}.
\begin{align}
    W(A) &\text{ is convex, closed and compact} \;,                     \label{eq:nr:convex}    \\
    W(\alpha \id + \beta A) &= \alpha + \beta W(A) \;,                  \label{eq:nr:linear}    \\
    W(A + B) &\subseteq W(A) + W(A) \;,                                 \label{eq:nr:subadd}    \\
    W(A^{\dagger}) &= W(A)^{\star} \;,                                  \label{eq:nr:dagger}    \\
    W(U A U^{\dagger}) &= W(A) \text{ for any unitary $U$} \;,          \label{eq:nr:unitary}   \\
    %\sigma(A) &\subseteq W(A) \;,                                       \label{eq:nr:spectrum}  \\
    \sigma(A) &\subseteq \convshull{A} \subseteq W(A) \;,               \label{eq:nr:convhull}  \\
    W(A) \text{ accretive } &\iff A+A^{\dagger} \text{ HPD } \;,        \label{eq:nr:accretive} \\
    A \text{ normal } &\iff \convshull{A} = W(A) \;.                    \label{eq:nr:normal}    %\\
    %\frac{1}{2} \sigma_{max}(A) &\leq \nrad{A} \leq \sigma_{max}(A) \;. \label{eq:nr:max:sing:val}
\end{align}
%Furthermore, for a normal operator \cref{eq:nr:convhull} is an equality and $A \text{ normal } \iff \convshull{A} = W(A)$.
The abbreviation HPD means Hermitian positive definite and $\sigma_{max}(A)$ is the largest singular value of $A$ or $\sigma_{max}(A) = \norm{A}_2$.
The sum of two sets in \cref{eq:nr:subadd} is the Minkowski sum; $X+Y = \{x+y \mid x \in X, \; y \in Y\}$.
\end{theorem}

\Cref{eq:nr:convex} is the famous Toeplitz-Hausdorff theorem~\cite{toeplitz1918algebraische,hausdorff1919wertvorrat}.
The numerical range of an operator on the fine grid is related to the numerical range of the coarse operator.

\begin{theorem}[Numerical range of coarse operators] \label{thm:numerical:range}
Assume $K$ to be an operator acting on the fine-grid lattice, $K \colon \vslattice \rightarrow \vslattice$ and $\coarse{K}$ a coarsening of $K$ acting on the coarse grid, $\coarse{K} \colon \coarse{\vslattice} \rightarrow \coarse{\vslattice}$.
Then
\begin{equation}
W(\coarse{K}) \subseteq W(K) \;.
\end{equation}
\end{theorem}

\begin{proof}
We look at the numerical range of the coarse operator
\begin{align*}
W({\coarse{K}})
&= \left\{ \coarse{\psi}^\dagger \coarse{K} \coarse{\psi} \; \middle| \; \coarse{\psi} \in \coarse{\vslattice} \; \text{and} \; \norm{\coarse{\psi}} = 1 \right\} \\
&= \left\{ (\restrictor \psi)^\dagger \coarse{K} (\restrictor \psi) \; \middle| \; \psi \in \vslattice \; \text{and} \; \norm{\restrictor \psi} = 1 \; \text{and} \; P \psi = \psi \right\} \\
&= \left\{ \psi^\dagger \prolongator \restrictor K \prolongator \restrictor \psi \; \middle| \; \psi \in \vslattice \; \text{and} \; \norm{\restrictor \psi} = 1 \; \text{and} \; P \psi = \psi \right\}
\end{align*}
where we've used that $\restrictor^\dagger = \prolongator$. We use that $P = P^\dagger = \prolongator \restrictor$ and $\norm{\restrictor \psi} = (\restrictor \psi)^\dagger \restrictor \psi = \psi^\dagger P \psi = \norm{\psi}$ if $P \psi = \psi$ and find
\begin{align*}
W({\coarse{K}})
&= \left\{ \psi^\dagger K \psi \; \middle| \; \psi \in \vslattice \; \text{and} \; \norm{\psi} = 1 \; \text{and} \; P \psi = \psi \right\} \\
&\subseteq \left\{ \psi^\dagger K \psi \; \middle| \; \psi \in \vslattice \; \text{and} \; \norm{\psi} = 1 \right\} \\
&= W({K}).
\end{align*}
\end{proof}

Importantly, this result holds for every imaginable coarsening of $K$.
Accretivity \cref{eq:nr:accretive}, is a very strong property, since according to the theorem above, every coarse operator is analytically invertible irrespective of the coarsening strategy.
% For an accretive fine-grid operator, we can even upper bound the condition number of coarse operators by
% \begin{equation}
% \kappa(\coarse{A}) \leq \frac{\nrad{A}}{ \inrad{A} } \,.
% \end{equation}
As we will see, the Dirac operator is \emph{not} accretive.

\subsection{Hermitian positive definite systems}

First, we will look at coarsenings of Hermitian positive definite (HPD) systems.
They form the easiest class of operators and we will analytically proof that for such an operator the condition numbers obey $\kappa(\coarse{A}) \leq \kappa(A)$, irrespective of how it is coarsened.

\begin{theorem}[Condition of coarse HPD operators] \label{thm:cond:hpd}
Assume $A$ to be a Hermitian positive definite operator acting on the fine-grid lattice, $A \colon \vslattice \rightarrow \vslattice$. Then the condition number of the coarse-grid operator $\coarse{A}$ is bound from above by the condition number of the fine-grid operator,
\begin{equation}
\kappa(\coarse{A}) \leq \kappa(A) \;.
\end{equation}
\end{theorem}

\begin{proof}
$A$ is Hermitian and positive definite, thus invertible: $\forall \lambda \in \sigma(A)$ we have $\lambda > 0$.
% We define the \df{numerical range} of an operator $A \colon \vslattice \rightarrow \vslattice$ as
% \begin{align} \label{eq:def:numerical:range}
% W(A) \coloneqq \left\{ \psi^\dagger A \psi \; \middle| \; \psi \in \vslattice \; \text{and} \; \norm{\psi} = 1 \right\}.
% \end{align}
The numerical range of $A$ is the interval in $\mathbb{R}_+$ delimited by the smallest and largest eigenvalue of $A$~\cite{gustafson1997numerical}, i.e. $W(A) = [ \lambda_{min}, \lambda_{max} ] \subset \mathbb{R}_+$.
% Then
% \begin{align*}
% W({\coarse{K}})
% &= \left\{ \coarse{\psi}^\dagger \coarse{K} \coarse{\psi} \; \middle| \; \coarse{\psi} \in \coarse{\vslattice} \; \text{and} \; \norm{\coarse{\psi}} = 1 \right\} \\
% &= \left\{ (\restrictor \psi)^\dagger \coarse{K} (\restrictor \psi) \; \middle| \; \psi \in \vslattice \; \text{and} \; \norm{\restrictor \psi} = 1 \; \text{and} \; P \psi = \psi \right\} \\
% &= \left\{ \psi^\dagger \prolongator \restrictor K \prolongator \restrictor \psi \; \middle| \; \psi \in \vslattice \; \text{and} \; \norm{\restrictor \psi} = 1 \; \text{and} \; P \psi = \psi \right\}
% \end{align*}
% where we've used that $\restrictor^\dagger = \prolongator$. We use that $P = P^\dagger = \prolongator \restrictor$ and $\norm{\restrictor \psi} = (\restrictor \psi)^\dagger \restrictor \psi = \psi^\dagger P \psi = \norm{\psi}$ if $P \psi = \psi$ and find
% \begin{align*}
% W({\coarse{K}})
% &= \left\{ \psi^\dagger K \psi \; \middle| \; \psi \in \vslattice \; \text{and} \; \norm{\psi} = 1 \; \text{and} \; P \psi = \psi \right\} \\
% &\subseteq \left\{ \psi^\dagger K \psi \; \middle| \; \psi \in \vslattice \; \text{and} \; \norm{\psi} = 1 \right\} \\
% &= W({K}).
% \end{align*}
Defining the smallest and largest eigenvalue of $\coarse{A}$ as $\mu_{min}$ and $\mu_{max}$ respectively, we use \cref{thm:numerical:range} and find $[ \mu_{min}, \mu_{max}] \subseteq [ \lambda_{min}, \lambda_{max} ]$, thus $\mu_{min} \geq \lambda_{min} > 0$ and $0 < \mu_{max} \leq \lambda_{max}$, which directly implies $\kappa(\coarse{A}) \leq \kappa(A)$, since all eigenvalues are positive. As a consequence, $\coarse{A}$ is HPD as well.
\end{proof}

\Cref{thm:cond:hpd} shows that with a HPD operator we do not need to care about how to coarsen.
Coarse operators are better conditioned in \emph{every} case.
Meaning that one can always coarsen $D^{\dagger} D$ instead of $D$ safely.
However, this operator is now next-to-nearest neighbor and its coarse variant $\coarse{D^{\dagger} D}$ is significantly less sparse and harder to implement efficiently.
Nevertheless this has been done in the field~\cite{Cohen:2011ivh,Boyle:2014rwa}.
%We continue with slightly more general operators.

\subsection{Symmetries}

We continue with properties that lead to symmetries in numerical ranges.

\begin{lemma}[Symmetry about the real axis] \label{lemma:nr:xsym}
Assume $K$ be a $\Gamma^{5}$-Hermitian operator, i.e. $K^{\dagger} = \Gamma^{5} K \Gamma^{5}$.
Then the numerical range of $K$ is symmetric about the real axis,
\begin{equation}
W(K) = W(K)^{\star} \iff \forall z \in W(K) \colon \bar{z} \in W(K) \;.
\end{equation}
\end{lemma}

\begin{proof}
We use \cref{eq:nr:dagger} as $W(K^{\dagger}) = W(K)^{\star}$ and \cref{eq:nr:unitary} subsequently $W(K^{\dagger}) = W(\Gamma^{5} K (\Gamma^{5})^{\dagger}) = W(K)$, since $\Gamma^{5}$ is unitary.
\end{proof}

By substituting $K=D$, the numerical ranges of the pure-Wilson \cref{eq:dirac:wilson} and Wilson-Clover Dirac operator \cref{eq:dirac:wilson:clover} are symmetric about the real axis, just as their eigenvalues are.

\begin{lemma}[Symmetry about the complex origin] \label{lemma:nr:osym}
Assume $K$ be an operator such that there exists a unitary $U$ such that $U K U^{\dagger} = -K$.
Then the numerical range of $K$ is symmetric about the complex origin,
\begin{equation}
W(K) = - W(K) \;.
\end{equation}
\end{lemma}

\begin{proof}
Straightforwardly $W(K) = W(U K U^{\dagger}) = W(-K) = -W(K)$, using first \cref{eq:nr:unitary}, then \cref{eq:nr:linear}.
\end{proof}

\begin{corollary}[Symmetry about the imaginary axis] \label{lemma:nr:ysym}
Assume $K$ be an operator as in \cref{lemma:nr:xsym,lemma:nr:osym}.
Then the numerical range of $K$ is symmetric about the imaginary axis,
\begin{equation}
W(K) = - W(K)^{\star} \;.
\end{equation}
\end{corollary}

\begin{proof}
Just concatenate lemmas \cref{lemma:nr:xsym,lemma:nr:osym}.
\end{proof}

\subsection{Dirac operators}

The hopping term of the Wilson (Clover) Dirac operator, defined as the off-diagonal part,
\begin{equation}
H(x,y) = \frac{1}{2} \sum_{\mu=}^{3}
\left\{
      U_{\mu}(x) (1 - \gamma_{\mu}) \delta_{x+\hat{\mu}, y}
    + U_{\mu}(x - \hat{\mu})^{\dagger} (1 + \gamma_{\mu}) \delta_{x-\hat{\mu}, y}
\right\} \;,
\end{equation}
obeys \cref{lemma:nr:xsym,lemma:nr:osym}:
Consider the parity operator that flips signs of odd lattice points only, $V(x,y) = \delta_{x,y} (-1)^{\sum_{\mu} x_{\mu}}$ satisfying $V^{2} = \id$ and $V^{\dagger} = V$.
With this the hopping term satisfies
\begin{equation}
V H V^{\dagger} = -H \;
\quad
\text{and}
\quad
\Gamma^{5} H \Gamma^{5} = H^{\dagger} \;.
\end{equation}
It follows that the numerical range of the hopping term $W(H)$ is symmetric about the real and imaginary axis as well as about the complex origin.
It is a superellipse with the complex origin as center point.
Furthermore, the pure Wilson operator \cref{eq:dirac:wilson} is just $D_W = (4+m) - H$.
It is thus the same superellipse, but shifted into the right-half plane with center point $\frac{1}{2 \kappa} = 4+m$ ($\kappa$ is the hopping parameter).
It is worth mentioning that $D_W$ is accretive if and only if the horizontal semi-axis $s$ of the superellipse satisfies $s < \frac{1}{2 \kappa}$.
%where the position shift operators are defined as $T_{\pm \mu}(x,y) = \delta_{x \pm \hat{\mu}, y}$.

Regarding the Wilson-Clover Dirac operator $D_{WC} = D_W + C$, \cref{eq:dirac:wilson:clover}, with the Hermitian,  indefinite Clover-term $C$, we know from \cref{thm:nr:properties}, that
\begin{equation}
W(D_{WC}) \subseteq  \frac{1}{2 \kappa} + W(H) + W(C) \;,
\end{equation}
and that $W(C)$ is the real line segment limited by its largest and smallest algebraic eigenvalues due to \cref{eq:nr:normal}.
For the numerical range $W(D_{WC})$, the superellipse of $W(D_{W})$ is thus squeezed or stretched along the real line by the values of $W(C)$, but the maximal imaginary extent is not altered\footnote{For stabilized Wilson fermions~\cite{Francis:2019muy} the exponential of the Clover term is taken instead. The Clover term -- now HPD -- is \emph{guaranteed} to move the numerical range further into the right-half plane, resulting in a higher potential (but no guarantee) for accretivity.}.

We have determined the numerical ranges of the Wilson and Wilson-Clover Dirac operators of a representative configuration of a dynamical periodic lattice in \cref{fig:nr:fine} (see \cref{ch:appendix:nr:ch:estimator} for an algorithm).
We can clearly see the symmetries discussed above.
On the same gauge config, the Clover-term stretches the numerical range into the left-half plane leading to $0 \in W(D_{WC})$ -- a loss of accretivity.
Then, coarse operators have the potential for zero or arbitrarily small eigenvalues.

\begin{figure}
\centering

\subfloat[Numerical range]{
    \includegraphics[width=0.47\textwidth]{\dir/img/nr/fine_nr}
    \label{fig:nr:fine}
}
\hfill
\subfloat[Spectral hull]{
    \includegraphics[width=0.47\textwidth]{\dir/img/sh/fine_sh}
    \label{fig:ch:fine}
}

\caption{
Boundary estimates of the numerical ranges (a) and spectral hulls (b) of the Wilson and Wilson-Clover Dirac operators.
}
\label{fig:fine}
\end{figure}

Numerical ranges of different coarse Wilson-Clover Dirac operators are plotted in \cref{fig:nr:coarse}.
\Cref{lemma:nr:xsym} suggests that the \caseXtwo{1}{2nc} cases, should not show symmetry about the real axis, because the coarse operator is not $\Gamma^{5}$-Hermitian.
It is barely visible in the plots, but those numerical ranges are indeed \emph{not} symmetric, whereas the ones in \cref{fig:nr:coarse:case2,fig:nr:coarse:case6} are.
For a numerical verification of the numerical range and spectral hull symmetries, please refer to \cref{ch:appendix:symmetries}.

Furthermore the numerical range around the complex origin is related to the low mode subspace.
Thus, we want this regime to be well covered.
Comparing \cref{fig:nr:coarse:case2,fig:nr:coarse:case6} which correspond to the strong chirality preserving cases, \caseXtwo{2sc}{4} in \cref{tab:spins}, the coarse numerical range around the origin is identical (see insets), suggesting that keeping all \num{4} spins does not give any more benefit than just keeping the \num{2} chiral ones.

\begin{figure}
\centering

\subfloat[\caseX{1}]{
    \includegraphics[width=0.47\textwidth]{\dir/img/nr/wc_Nc20_Ns1_nr}
    \label{fig:nr:coarse:case1}
}
\hfill
\subfloat[\caseX{2sc}]{
    \includegraphics[width=0.47\textwidth]{\dir/img/nr/wc_Nc20_Ns2_nr}
    \label{fig:nr:coarse:case2}
}

\subfloat[\caseX{2nc}]{
    \includegraphics[width=0.47\textwidth]{\dir/img/nr/wc_Nc20_P02P13_nr}
    \label{fig:nr:coarse:case4}
}
\hfill
\subfloat[\caseX{4}]{
    \includegraphics[width=0.47\textwidth]{\dir/img/nr/wc_Nc20_Ns4_nr}
    \label{fig:nr:coarse:case6}
}

\caption{
Numerical range boundary estimates of coarse and fine Wilson-Clover Dirac operators.
All subspaces were generated with $\Nc=20$ low modes of $\Gamma^{5} D$.
The panels correspond to spin projectors in \cref{tab:spins}.
Every panel shows different coarsening block sizes, where ``LMA'' corresponds to low-mode averaging, ``fine'' to the fine-grid operator and the MG aggregate block sizes are indicted in the labels.
}
\label{fig:nr:coarse}
\end{figure}

\section{Spectral hull}

We continue with studying the spectral hull, $\convshull{A}$, for fine and coarse Dirac operators.
According to \cref{thm:nr:properties}, the spectral hull is always contained in the numerical range of an operator.
For a non-normal operator they do not coincide, comparing the areas of the numerical range and the spectral hull can be measure for non-normality.

We have determined the spectral hulls of the Wilson and Wilson-Clover Dirac operators \cref{fig:ch:fine} (see \cref{ch:appendix:nr:ch:estimator} for an algorithm).
Both are positively stable (spectral hull is in the open right half plane).
We see that the spectrum of the Wilson-Clover operator underwent the same squeezing caused by adding the Clover-term, similar to the numerical range.
Notable is that the free (gauge fields set to unity) Wilson operators numerical range and spectral hull coincide, except for the rightmost piece.
Eigenvectors associated to the extremal eigenvalues lying on the boundary of the numerical range are thus orthogonal to all other eigenvectors, i.e. the operator restricted to these eigenvectors is normal, $D^\dagger D \xi = D D^\dagger \xi$.

Spectral hulls of different coarse Wilson-Clover Dirac operators are plotted in \cref{fig:ch:coarse}.
If chirality is strongly preserved, we observe spectral hull inclusion from coarser to finer grids,
\begin{equation}
\convshull{\coarse{D}_{\text{LMA}}} \subseteq
\convshull{\coarse{D}_{6^{4}}} \subseteq
\convshull{\coarse{D}_{4^{4}}} \subseteq
\convshull{D_{\text{fine}}} \,.
\end{equation}
This is a strong result, but unfortunately it does not translate to $\kappa(\coarse{D}) \leq \kappa(D)$ yet, because $D$ is non-normal.
% This is the strongest result of this section, because it makes coarse operators well conditioned and immediately translates to condition numbers,
% %and we immediately find $\kappa(\coarse{D}) \leq \kappa(D)$.
% \begin{equation}
% \kappa(\coarse{D}_{\text{LMA}}) \leq
% \kappa(\coarse{D}_{6^{4}}) \leq
% \kappa(\coarse{D}_{4^{4}}) \leq
% \kappa(D_{\text{fine}}) \,.
% \end{equation}
In \cref{fig:ch:coarse:case1,fig:ch:coarse:case4}, spectral hull inclusion is not the case: in both panels all \num{3} coarse operators show eigenvalues in magnitude smaller than the smallest fine one.
This poses clear problems for numerical inversion.

To summarize, we collected condition numbers of relevant fine and coarse operators in \cref{tab:condition}.
Clearly the problem of spurious eigenvalues is not as severe for coarsenings of the non-Hermitian Dirac operator as compared to the Hermitian version, where the condition number explodes and thus quickly becomes numerically inaccessible.
It is the smallest magnitude singular value (fourth column) responsible for the operators to become ill-conditioned as the highest one is capped by the maximal singular value on the fine grid.
Boldface quantities coincide with the smallest singular value of the fine grid up to determination precision, whereas gray quantities fall below it.

This shows that weak chirality preservation alone is not enough to guarantee well conditioned systems.
For the strong chirality preserving cases on the other hand, the smallest singular value is capped by the smallest one on the fine grid.
Empirically, we find that strong chirality preservation leads to preservation of extremal eigen- and singular values.
According to \cref{lemma:svd:coarse} in that case, the singular values of coarse Hermitian and non-Hermitian operators are equal, and by this their condition numbers.
Spectra for Hermitian operators are easier accessible numerically.
For that reason, we further investigated a larger chunk of the low lying spectrum of the Hermitian Dirac operator in the next section.

\begin{figure}
\centering

\subfloat[\caseX{1}]{
    \includegraphics[width=0.47\textwidth]{\dir/img/sh/wc_Nc20_Ns1_sh}
    \label{fig:ch:coarse:case1}
}
\hfill
\subfloat[\caseX{2sc}]{
    \includegraphics[width=0.47\textwidth]{\dir/img/sh/wc_Nc20_Ns2_sh}
    \label{fig:ch:coarse:case2}
}

\subfloat[\caseX{2nc}]{
    \includegraphics[width=0.47\textwidth]{\dir/img/sh/wc_Nc20_P02P13_sh}
    \label{fig:ch:coarse:case4}
}
\hfill
\subfloat[\caseX{4}]{
    \includegraphics[width=0.47\textwidth]{\dir/img/sh/wc_Nc20_Ns4_sh}
    \label{fig:ch:coarse:case6}
}

\caption{
Spectral hull boundary estimates of coarse and fine Wilson-Clover Dirac operators.
This plots series corresponds to \cref{fig:nr:coarse}.
% The subspaces were generated with $\Nc=20$ low modes of $\Gamma^{5} D$.
% The panels correspond to spin projectors in \cref{tab:spins}.
% Every panels shows different coarsening block sizes, where ``no blocking'' corresponds to LMA and ``fine'' to the fine-grid operator.
Note that the inset x-axis is log-scale, meaning that the convex hull might not look convex everywhere, because the logarithm is a non-linear transformation.
}
\label{fig:ch:coarse}
\end{figure}

% \begin{table}
% \begin{tabular}{
% llc|
% rrr
% r
% }
% \toprule
% Operator $A$&
% Grid &
% \makecell{Chirality\\preservation}
% & $\sigma_{min}(A)$
% & $\sigma_{max}(A)$
% %& $\nrad{A}$
% & $\kappa(A)$
% & $\kappa(A)2$
% \\
% \midrule
% % $D_W, Q_W$ (free) & fine & -        & \nt{min}{x}
% %                                     & \nt{max}{x}
% %                                     & \nt{7.67}{7.667862627353456}
% %                                     & \nt{kappa}{x} \\
% % $D_W, Q_W$    & fine     & -        & \nt{0.435}{4.3549788925848770e-01}
% %                                     & \nt{7.17}{7.1730788452329097e+00}
% %                                     & \nt{7.11}{7.1108396881408895}
% %                                     & \nt{$16.5(0)$}{1.6470984181912677e+01} \\
% $D_{WC}, Q_{WC}$ & fine  & -        & \nt{5.55e-3}{5.5481037267318730e-03}
%                                     & \nt{7.65}{7.6462421529480125e+00}
%                                     %& \nt{7.49}{7.491162918196567}
%                                     & \nt{$1380(20)$}{1.3781721700888340e+03}
%                                     & 1380
% \\
% % \midrule
% % $\coarse{D}_{WC}$ & LMA  & no       & \ut{4.18e-3}{4.1796137146807334e-03}
% %                                     & \nt{1.60e-2}{1.6013853253362759e-02}
% %                                     %& \nt{1.55e-2}{0.015522532435122433}
% %                                     & \nt{$3.8(1)$}{3.8314194436473188e+00}
% %                                     & 3.8(1)
% % \\
% % (\caseX{1})    & MG $6^4$ & no       & \ut{4.44e-3}{4.4422207902965645e-03}
% %                                     & \nt{1.46}{1.4591063032306548e+00}
% %                                     %& \nt{1.45}{1.4545971649705798}
% %                                     & \nt{$328(8)$}{3.2846325568010417e+02}
% %                                     & 328(8)
% % \\
% %               & MG $4^4$ & no       & \ut{4.54e-3}{4.5403408953169242e-03}
% %                                     & \nt{2.04}{2.0392773188336357e+00}
% %                                     %& \nt{2.03}{2.028765960303842}
% %                                     & \nt{$450(10)$}{4.4914630109316721e+02}
% %                                     & 450(10)
% % \\
% % \midrule
% % $\coarse{Q}_{WC}$ & LMA  & no       & \nt{5.55e-3}{5.548102e-03}
% %                                     & \nt{2.31e-2}{2.312314e-02}
% %                                     %& \nt{2.31e-2}{2.312314e-0,}
% %                                     & \nt{$4.16(7)$}{4.1677568292724}
% %                                     & 4.16(7)
% % \\
% % (\caseX{1})    & MG $6^4$ & no      & \ut{6.71e-4}{6.7129933376974750e-04 +/- 9.8931927945458135e-07}
% %                                     & \nt{1.08}{1.0840794144164108e+00 +/- 7.3226048798409186e-07}
% %                                     %& \nt{1.08}{1.0840794144164108e+00 +/- 7.3226048798409186e-07}
% %                                     & \nt{$1615(2)$}{1.6148971999251899e+03 +/- 2.3799354798435797e+00}
% %                                     & 1615(2)
% % \\
% %               & MG $4^4$ & no       & \ut{4.08e-5}{4.0847606195743720e-06 +/- 9.9131038427911789e-08}
% %                                     & \nt{1.58}{1.5680603876706616e+00 +/- 8.9749395241456465e-08}
% %                                     %& \nt{1.58}{1.5680603876706616e+00 +/- 8.9749395241456465e-08}
% %                                     & \nt{$38400(9300)$}{3.8388060738649894e+05 +/- 9.3162089010327891e+03}
% %                                     & 38400(9300)
% % \\
% % \midrule
% % $\coarse{D}_{WC}, \coarse{Q}_{WC}$ & LMA  & strong
% %                                     & \nt{5.55e-3}{5.5482402034070137e-03}
% %                                     & \nt{1.02}{1.0239883312989366e+00}
% %                                     %& \nt{0.926}{0.9255000322214896}
% %                                     & \nt{$185(3)$}{1.8456092269944170e+02}
% % \\
% % (\caseX{2sc})  & MG $6^4$ & strong   & \nt{5.55e-3}{5.5481098688822785e-03}
% %                                     & \nt{2.03}{2.0329543697857719e+00}
% %                                     %& \nt{2.00}{2.003202952499008}
% %                                     & \nt{$366(6)$}{3.6642287514672643e+02}
% % \\
% %               & MG $4^4$ & strong   & \nt{5.55e-3}{5.5481056289286559e-03}
% %                                     & \nt{2.59}{2.5881098949207302e+00}
% %                                     %& \nt{2.56}{2.56364960063965}
% %                                     & \nt{$466(8)$}{4.6648533175466895e+02}
% % \\
% % \midrule
% % $\coarse{D}_{WC}, \coarse{Q}_{WC}$ & LMA  & weak
% %                                     & \nt{6.23e-2}{6.2289915389005336e-02}
% %                                     & \nt{1.67}{1.6752832371696156e+00}
% %                                     %& \nt{1.67}{1.6752832371696156e+00}
% %                                     & \nt{26.9}{2.6894935186656497e+01}
% % \\
% % (\caseX{2wc})  & MG $6^4$ & weak     & \ut{2.40e-4}{-2.400382E-04 +/- 4.405E-09}
% %                                     & \nt{2.80}{2.7986287048240293e+00 + -3.6042141952009586e-17i}
% %                                     %& \nt{2.80}{2.7986287048240293e+00 + -3.6042141952009586e-17i}
% %                                     & \nt{$11659(1)$}{11659.097197129578}
% % \\
% %               & MG $4^4$ & weak     & \ut{1.62e-5}{1.6168019008400799e-05 +/- 9.5167509651452973e-13}
% %                                     & \nt{3.45}{3.4466899646585309e+00 +/- 7.1179649138649085e-07}
% %                                     %& \nt{nrad}{x}
% %                                     & \nt{\num{213179(1)}}{2.1317948493675404e+05 +/- 4.5778291614500401e-02}
% % \\
% % \midrule
% % $\coarse{D}_{WC}, \coarse{Q}_{WC}$ & LMA  & PpP2P3
% %                                     & \nt{min}{x}
% %                                     & \nt{max}{x}
% %                                     & \nt{nrad}{x}
% %                                     & \nt{kappa}{x} \\
% %               & MG $6^4$ & PpP2P3   & \nt{min}{5.55e-3}
% %                                     & \nt{max}{2.7662710934964965}
% %                                     & \nt{nrad}{x}
% %                                     & \nt{kappa}{x} \\
% %               & MG $4^4$ & PpP2P3   & \nt{min}{x}
% %                                     & \nt{max}{x}
% %                                     & \nt{nrad}{x}
% %                                     & \nt{kappa}{x} \\
% % \midrule
% % $\coarse{D}_{WC}, \coarse{Q}_{WC}$ & LMA  & P0P2
% %                                     & \nt{min}{x}
% %                                     & \nt{max}{x}
% %                                     & \nt{nrad}{x}
% %                                     & \nt{kappa}{x} \\
% %               & MG $6^4$ & P0P2     & \nt{0.274}{2.7363581934309827e-01}
% %                                     & \nt{2.45}{2.4562335794482659e+00}
% %                                     & \nt{nrad}{x}
% %                                     & \nt{8.99}{8.9762867498297716e+00} \\
% %               & MG $4^4$ & P0P2     & \nt{0.256}{2.5563392568993265e-01}
% %                                     & \nt{3.05}{3.0528947892804035e+00}
% %                                     & \nt{nrad}{x}
% %                                     & \nt{11.9}{1.1942447705408892e+01} \\
% \bottomrule
% \end{tabular}
% \caption{
% Extremal singular values $\sigma_{min,max}(A)$ and condition numbers $\kappa(A)$ for some coarse and fine, Hermitian and non-Hermitian Dirac operators.
% $D_{WC}$ indicates the Wilson-Clover Dirac operator and $Q$ is the Hermitian one $Q = \Gamma^{5} D$.
% For all coarsenings the $\Nc = 20$ lowest modes of $Q_{WC}$ were taken.
% The underlined quantities are smallest singular values smaller than the fine grid one.
% Associated operators are numerically problematic.
% }
% \label{tab:condition}
% \end{table}

\newcommand{\nt}[2]{\num{#1}}
\newcommand{\ut}[2]{\textcolor{gray}{\num{#1}}}
\newcommand{\st}[2]{\textbf{\num{#1}}} %\newcommand{\st}[2]{#1$^\star$}
\begin{table}
\begin{tabular}{
llc|
S[tight-spacing=true, uncertainty-mode=compact-marker, math-rm=\mathrm, mode=text, detect-weight=true, detect-family=true, round-mode=figures, round-precision=2, scientific-notation=true, output-exponent-marker=e]
S[tight-spacing=true, uncertainty-mode=compact-marker, math-rm=\mathrm, mode=text, detect-weight=true, detect-family=true, round-mode=figures, round-precision=2]
S[tight-spacing=true, uncertainty-mode=compact-marker, math-rm=\mathrm, mode=text, detect-weight=true, detect-family=true]
}
\toprule
Operator $A$&
Grid &
\makecell{Chirality\\preservation} &
\multicolumn{1}{c}{$\sigma_{min}(A)$} &
\multicolumn{1}{c}{$\sigma_{max}(A)$} &
\multicolumn{1}{c}{$\kappa(A)$} \\
\midrule
$D_{WC}, Q_{WC}$ & fine  & -        & \st{5.55e-3}{5.5481037267318730e-03}
                                    & \nt{7.65}{7.6462421529480125e+00}
                                    %& \nt{7.49}{7.491162918196567}
                                    & 1380 % 1.3781721700888340e+03
\\
\midrule
$\coarse{D}_{WC}$ & LMA  & no       & \ut{4.18e-3}{4.1796137146807334e-03}
                                    & \nt{0.0160}{1.6013853253362759e-02}
                                    %& \nt{1.55e-2}{0.015522532435122433}
                                    & \nt{3.8(1)}{3.8314194436473188e+00}
\\
\caseX{1}     & MG $6^4$ & no       & \ut{4.44e-3}{4.4422207902965645e-03}
                                    & \nt{1.4591063032306548e+00}{1.4591063032306548e+00}
                                    %& \nt{1.45}{1.4545971649705798}
                                    & \nt{328(8)}{3.2846325568010417e+02}
\\
              & MG $4^4$ & no       & \ut{4.54e-3}{4.5403408953169242e-03}
                                    & \nt{2.0392773188336357e+00}{2.0392773188336357e+00}
                                    %& \nt{2.03}{2.028765960303842}
                                    & \nt{450(10)}{4.4914630109316721e+02}
\\
\midrule
$\coarse{Q}_{WC}$ & LMA  & no       & \st{5.55e-3}{5.548102e-03}
                                    & \nt{0.0231}{2.312314e-02}
                                    %& \nt{2.31e-2}{2.312314e-0,}
                                    & \nt{4.16(7)}{4.1677568292724}
\\
\caseX{1}     & MG $6^4$ & no       & \ut{6.71e-4}{6.7129933376974750e-04 +/- 9.8931927945458135e-07}
                                    & \nt{1.0840794144164108e+00}{1.0840794144164108e+00 +/- 7.3226048798409186e-07}
                                    %& \nt{1.08}{1.0840794144164108e+00 +/- 7.3226048798409186e-07}
                                    & \nt{1615(2)}{1.6148971999251899e+03 +/- 2.3799354798435797e+00}
\\
              & MG $4^4$ & no       & \ut{4.08e-5}{4.0847606195743720e-06 +/- 9.9131038427911789e-08}
                                    & \nt{1.5680603876706616e+00}{1.5680603876706616e+00 +/- 8.9749395241456465e-08}
                                    %& \nt{1.58}{1.5680603876706616e+00 +/- 8.9749395241456465e-08}
                                    & \nt{38400(9300)}{3.8388060738649894e+05 +/- 9.3162089010327891e+03}
\\
\midrule
$\coarse{D}_{WC}, \coarse{Q}_{WC}$  & LMA  & strong
                                    & \st{5.55e-3}{5.5482402034070137e-03}
                                    & \nt{1.0239883312989366e+00}{1.0239883312989366e+00}
                                    %& \nt{0.926}{0.9255000322214896}
                                    & \nt{185(3)}{1.8456092269944170e+02}
\\
\caseX{2sc}   & MG $6^4$ & strong   & \st{5.55e-3}{5.5481098688822785e-03}
                                    & \nt{2.0329543697857719e+00}{2.0329543697857719e+00}
                                    %& \nt{2.00}{2.003202952499008}
                                    & \nt{366(6)}{3.6642287514672643e+02}
\\
              & MG $4^4$ & strong   & \st{5.55e-3}{5.5481056289286559e-03}
                                    & \nt{2.5881098949207302e+00}{2.5881098949207302e+00}
                                    %& \nt{2.56}{2.56364960063965}
                                    & \nt{466(8)}{4.6648533175466895e+02}
\\
\midrule
$\coarse{D}_{WC}, \coarse{Q}_{WC}$  & LMA  & weak
                                    & \nt{6.23e-2}{6.2289915389005336e-02}
                                    & \nt{1.6752832371696156e+00}{1.6752832371696156e+00}
                                    %& \nt{1.67}{1.6752832371696156e+00}
                                    & \nt{26.9}{2.6894935186656497e+01}
\\
\caseX{2wc}   & MG $6^4$ & weak     & \ut{2.40e-4}{-2.400382E-04 +/- 4.405E-09}
                                    & \nt{2.7986287048240293e+00}{2.7986287048240293e+00 + -3.6042141952009586e-17i}
                                    %& \nt{2.80}{2.7986287048240293e+00 + -3.6042141952009586e-17i}
                                    & \nt{11659(1)}{11659.097197129578}
\\
              & MG $4^4$ & weak     & \ut{1.62e-5}{1.6168019008400799e-05 +/- 9.5167509651452973e-13}
                                    & \nt{3.4466899646585309e+00}{3.4466899646585309e+00 +/- 7.1179649138649085e-07}
                                    %& \nt{nrad}{x}
                                    & \nt{213179(1)}{2.1317948493675404e+05 +/- 4.5778291614500401e-02}
\\
\bottomrule
\end{tabular}
\caption{
Extremal singular values $\sigma_{min,max}(A)$ and condition numbers $\kappa(A)$ for some coarse and fine, Hermitian and non-Hermitian Dirac operators.
$D_{WC}$ indicates the Wilson-Clover Dirac operator and $Q$ is the Hermitian one $Q = \Gamma^{5} D$.
For all coarsenings the $\Nc = 20$ lowest modes of $Q_{WC}$ were taken.
Gray quantities indicate smallest singular values smaller than the fine grid one.
Associated operators are numerically problematic.
}
\label{tab:condition}
\end{table}

\section{Numerical eigenvalue study}

%To support our findings until now,
We performed a spectral study of a Hermitian Wilson-Clover Dirac operator $\Gamma^{5} D$.
In this study, we generated multigrid subspaces in two different ways:
\begin{enumerate}
\item By naive coarsening corresponding to the \caseX{1} case in \cref{tab:spins} with $N_s=1$ remaining spins on the coarse grid.
\item By strongly preserving chirality corresponding to the \caseX{2sc} case with $N_s=2$ coarse spins.
\end{enumerate}

\begin{figure}
\centering
\includegraphics[width=1.0\linewidth]{\dir/img/eigenvalues_Nc20}
\caption{
Study of low lying spectra of coarse and fine Hermitian Dirac operators.
Lower left: eigenvalue magnitude versus mode number of the \num{100} lowest modes, lower right: eigenvalue magnitudes between fine and different coarse operators for comparison, and top panel: zoom of lower right of the interesting region.
The three operators considered are fine (blue), coarse \pcaseX{1} (yellow) and coarse \pcaseX{2sc} (green), c.f. \cref{tab:spins}.
\takenfull
}
\label{fig:chirality:spectrum}
\end{figure}

\Cref{fig:chirality:spectrum} presents the results.
In blue the first \num{100} actual smallest-magnitude eigenvalues of the fine-grid Hermitian operator $Q = \Gamma^5 D$ are plotted.
The $\Nc=20$ lowest ones where taken to generate the coarse subspace for both scenarios.
For both coarse operators we also plotted their first \num{100} smallest-magnitude eigenvalues in yellow and green.
In the lower left panel their magnitudes are plotted, the top panel is a zoom to the relevant region of the lower right panel, which shows a tick for every eigenvalue at its magnitude for comparison.
Every fine-grid eigenvalue has grey vertical line drawn for eye guidance.
The shaded region denotes the first \num{20} fine-grid modes for which we expect perfect overlap by construction for both coarse subspaces as can be confirmed in the zoomed panel.

When coarsening all spins (yellow), we observe coarse eigenvalues lower than the lowest fine one as well as other spurious ones in the purple region.
These eigenvalues are responsible for ill-conditioned coarse systems and the slowing down of multigrid as a preconditioner as well as multigrid as variance reduction method introduced above.
On the other hand, when chiral properties are preserved (green), we cannot find a single eigenvalue in the critical purple region that is not an inherited one from the fine-grid.
Coarse eigenvalues above the purple region are distorted on both cases as expected, but they do not have impact in the condition.

\section{Numerical variance study}

The preserved chiral properties do not only have impact on the spectrum, but also the variance contribution of coarse subspaces.
\Cref{fig:chirality:variance} shows the impact of coarse chirality on the variances.
We can clearly see a tremendous benefit from the strong chirality preservation as opposed to its absence.
Variance contributing to the large distance regime therefore originates from both chiral sectors equally.
The yellow data corresponds to the \caseX{1} case from \cref{tab:spins}, whereas the green data to the \caseX{2sc} case.
Even though the \caseX{2sc} subspace is twice as large as the \caseX{1} one, it is a tradeoff absolutely worth doing, not only for the coarse condition, but also for its variance contribution.
\begin{figure}
\centering
\includegraphics[width=1.0\linewidth]{\dir/img/chirality}
\caption{
Variance contribution of the \Ln{0}-term if all spin degrees of freedom are coarsened \pcaseX{1} (yellow) as compared to strong chirality preservation \pcaseX{2sc} (green).
Variance contributions of both LMA (left) and MG LMA (right) profit from these considerations, although for LMA, the benefit is more subtle.
\takenfull
}
\label{fig:chirality:variance}
\end{figure}


%\worktodo{plots with convex spectral hull, Ns2 Ns4 show spectral hull inclusion, Ns1 not!}
%\worktodo{table with areas and symmetry-metric.}

% \section{Simple operators}

% \textcolor{gray}{
% \begin{theorem}[Condition of coarse normal, positively stable operators] \label{thm:cond:normal:pos:stable}
% Assume $K$ to be a normal, $\Gamma^{5}$-Hermitian, positively stable ($\forall \lambda \in \sigma(B) \; : \; \Re{\lambda} > 0$) operator acting on the fine-grid lattice, $K \colon \vslattice \rightarrow \vslattice$, whose eigenvalue with smallest magnitude is real.
% Then the condition number of the coarse-grid operator $\coarse{K}$ is bound from above by the condition number of the fine-grid operator,
% \begin{equation}
% \kappa(\coarse{K}) \leq \kappa(K) \;.
% \end{equation}
% \end{theorem}
% }

% \textcolor{gray}{
% \begin{proof}
% If $K$ is normal, its numerical range equals to the convex hull of its spectrum, $W(K) = \text{conv}\{\sigma(K)\}$, which in turn is symmetric about the real axis using \cref{lemma:nr:xsym}.
% Thus $\sigma(\coarse{K}) \subseteq \text{conv}\{\sigma(K)\}$ and the element of smallest magnitude of $\text{conv}\{\sigma(K)\}$ is the real part of the eigenvalue of $K$ with smallest magnitude.
% By assumption this eigenvalue is real.
% Its largest element in magnitude is always smaller or equal to the eigenvalue of $K$ with largest magnitude.
% Therefore $\kappa(\coarse{K}) \leq \kappa(K)$.
% \end{proof}
% }

% \textcolor{gray}{
% If the smallest magnitude eigenvalue of $K$ is not real, but complex in general, then the smallest magnitude element of $\text{conv}\{\sigma(K)\}$ is given by the real part of the smallest magnitude eigenvalue instead.
% We then cannot guarantee anymore that the smallest magnitude eigenvalue of $\coarse{K}$ is not smaller than the one of $K$.
% At least, we have the guarantee that $\coarse{K}$ is invertible, because $0 \not\in \text{conv}\{\sigma(K)\}$, a consequence of $K$ being positively stable.
% }

% \section{Final conjecture}

% \textcolor{gray}{
% We finish this chapter with a conjecture, motivated by the numerical and analytical studies above, from which we believe it is a sufficient condition for a well-behaved coarse-grid Dirac operator spectrum, suitable for coarse grid inversions as required by multigrid low-mode averaging.
% \begin{conj}
% Assume $K$ to be a positively stable, $\Gamma^{5}$-Hermitian operator acting on the fine-grid lattice, $K \colon \vslattice \rightarrow \vslattice$, not necessarily normal.
% With Hermitian $P$ and $\Gamma^{5}$, we conjecture that if
% \begin{equation} \label{eq:conj:chi}
% \exists B \in \ggrp{GL}{N_s \Nc, \mathbb{C}} \; \colon \; \rho^5 = B \left( \chi^{5} \otimes \id_{\coarse{S}} \otimes \id_{\coarse{\idxcolor}} \right) B^{-1}
% \end{equation}
% for a change-of-basis matrix $B$, $\id_{\coarse{S}}$ is the identity of dimension either $1$ or $2$,
% % and it holds
% % \begin{equation} \label{eq:conj:modes}
% % \forall i \; P \evec_i = \evec_i \;,
% % \end{equation}
% then chirality is preserved on the coarse lattice and
% \begin{equation}
% \kappa(\coarse{K}) \leq \kappa(K) \;.
% \end{equation}
% \end{conj}
% }

% \textcolor{gray}{
% Additionally we demand invariance of the $N_c$ lowest eigenmodes
% \begin{equation} \label{eq:conj:modes}
% \forall i \; P \evec_i = \evec_i \;,
% \end{equation}
% since this is a requirement coming from low-mode averaging.
% %The invariance of the eigenmodes \cref{eq:conj:modes} is a requirement coming from low-mode averaging.
% If not satisfied the subspace will not include the exact low modes resulting in a degraded variance reduction compared to low-mode averaging.
% The particular form of the coarse chirality operator \cref{eq:conj:chi} on the other hand fulfills the statements in \cref{lemma:chirality:preservation:equiv,lemma:chirality:preservation:implications,lemma:svd:coarse,lemma:evals:coarse} while preserving the trace.
% }

\section{Summary}
\label{sec:chirality:summary}

%\worktodo{Main message: if chirality is strongly preserved this is a sufficient condition that implies kappa(coarseD) leq kappa(D), weak preservation is not enough!}

%We found a sufficient condition for well conditioned coarse operators.
%The studies

The numerical range studies revealed important information about Wilson-type Dirac operators.
Critically, the Wilson-Clover operator is not accretive, even more so for physical masses, i.e. the operator itself, but also its coarse variants have the potential for zero or arbitrarily small modes in their spectra.
%We thus need a coarsening strategy that .
Weak chirality preservation $[P, \Gamma^{5}]=0$ is responsible for the inheritance of many properties, amongst others equal singular values for non-Hermitian $\coarse{K}$ and Hermitian $\coarse{\Gamma^{5} K}$ coarsenings.
Empirically we observed spectral hull inclusion and overlap of the $\Nc$ lowest modes for coarsening schemes that strongly preserve chiral properties, no so for weak chiral, non-chiral or no spin preservation at all.
Therefore it is the chiral degree of freedom alone, that is responsible for inheriting well behaved spectra.
%Spectral hull inclusion guarantees well-conditioned coarse operators, which is crucial for MG LMA and multigrid as a preconditioner to be beneficial.

%Chirality on the coarse subspaces is an important concept. %, and critical for MG LMA and multigrid as a preconditioner to be beneficial.
%We have seen multiple ways to achieve this, some of which are equivalent in their outcome.
%We have studied the shapes of numerical ranges and spectral hulls of various coarse and fine Dirac operators.
%The outcome was that for coarse operator spectra to behave well, we need to explicitly preserve fine-grid chirality.
%The non-chiral spin degrees of freedom, if not preserved together with the chiral ones, still show smaller eigenvalues and lead to ill-conditioned coarse systems.
%Also chiral indices have to be treated symmetrically to preserve the chiral operator trace.
%Furthermore it is evident that the low-modes have to be perfectly represented in the subspace, else the variance reduction will be degraded.

%We systemically relaxed assumptions to the operator, still upholding this guarantee by studying their spectra. 
%On this way, preserving spectral properties on coarse grids is crucial.
For Hermitian positive definite systems, we are guaranteed to have better conditioned coarse systems, no matter how we coarsen.
We believe that careful study of numerical range theory on other lattice discretizations reveals further algorithmic and theoretical insights.

Concluding, the main finding of this section is:\\
\textbf{
For an invertible $\Gamma^{5}$-Hermitian operator $K$ acting on the lattice,
if chirality is strongly preserved, i.e. if $[P, \Gamma^{5}] = 0$ and $\tr(\coarse{\Gamma^5}) = 0$, then coarse operators $\coarse{K}$ satisfy
\begin{equation}
\kappa(\coarse{K}) \leq \kappa(K) \;.
\end{equation}
}
