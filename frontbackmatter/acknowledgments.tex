%*******************************************************
% Acknowledgments
%*******************************************************
\pdfbookmark[1]{Acknowledgements}{acknowledgements}

\bigskip

\begingroup
\let\clearpage\relax
\let\cleardoublepage\relax
\let\cleardoublepage\relax
\chapter*{Acknowledgements}

\def\thanks#1{%
\begingroup
\leftskip1em
\noindent #1
\par
\endgroup
}

%\worktodo{
%* Marina who has introduced me to this vast research field of possibilities
%* Schulthess introduced me to marina for my master thesis and served as my second supervisor
%* ZH people
%* RCstar people
%* Cheryl for her continuous support and for fighting through an early version of this document (voluntarily!).
%* countless nights with coffee
%* my cats for chasing the mouse cursor or sitting directly onto my keyboard (footnote: all errors in this document arose through this, I attribute all errors in this document to them) or zoom-bombing me repeatedly
%}

%\worktodo{I would like to thank \dots}

First and foremost, I would like to thank my supervisor, Prof. Marina Marinković, for her continuous guidance, patience and for introducing me to this vast research field with its many possibilities.
I appreciate the freedom and trust (which I think I have not deserved) she gave me to explore my own ideas.

I thank Prof. Thomas Schulthess for introducing me to Marina during my master thesis and for serving as my second supervisor.

The people from the research group in Zürich, I would like to thank for their inputs and constructive feedback, specially Tim Harris for countless discussions about noise and for showing me repeatedly that variances of diagrams are diagrams themselves\footnote{Don't tell him, but I still don't entirely get it.}.

I thank the members of the \RCstar collaboration for accepting me as one of their own, giving me the opportunity to integrate my research into the broader perspective of its scientific goals, for enduring my way-too-long presentations on various topics and for tolerating me being annoying about software development aspects.

My girlfriend Cheryl supported me all the way through, I am grateful for that.
She made it through an early (even less readable) version of this document\footnote{Voluntarily!}.

This work would not have been possible without my four cats;
%Mössssssiöö Schrödinger, Dame July, Birchermüesli and Tazzzzebüsi
my emotional support animals.
I thank them for distracting me, chasing my mouse cursor, zoom-bombing me or sitting directly on my keyboardddddddddddddddd\footnote{Undoubtedly, this led to some of my best results, but I attribute all errors in this document to them alone.}.
They were my constant companions throughout countless coffee-fueled days and nights.

I am grateful to Richard Brower, Kate Clark, Carlton De Tar, Aida El-Khadra, Leonardo Giusti, Steven Gottlieb, Bartosz Kostrzewa, Simon Kuberski, Christoph Lehner, Martin Lüscher, Mathias Wagner and Evan Weinberg for insightful discussions at conferences, workshops or visits.

The support for the PASC 2021-2024 project ``Efficient QCD+QED Simulations with openQ*D software''~\cite{online:pasc2021} is gratefully acknowledged.

I acknowledge the access to Piz Daint and Daint Alps at the Swiss National Supercomputing Centre, Switzerland under the ETHZ’s share with the project IDs c21, ch16, eth8, go21 and s1196.

I gratefully acknowledge the Gauss Centre for Supercomputing e.V. (www.gauss-centre.eu) for funding this project by providing computing time on the GCS Supercomputer JUWELS~\cite{juwels} at Jülich Supercomputing Centre (JSC). 

I acknowledge the EuroHPC Joint Undertaking for awarding this project access to the EuroHPC supercomputer LUMI, hosted by CSC (Finland) and the LUMI consortium through a EuroHPC Regular Access call.

I acknowledge the EuroHPC Joint Undertaking for awarding this project access to the EuroHPC supercomputer LEONARDO, hosted by CINECA (Italy) and the LEONARDO consortium through an EuroHPC Regular Access call.

% \worktodo{
% This work was supported by the  Platform for Advanced Scientific Computing (PASC) project "name of the project". \cite{online:pasc2021}"
% * old daint
% * daint alps
% * LUMI-C
% * JUWELS
% * Leonardo
% }

Finally, I want to explicitly clarify how and to what extent I have used generative AI technologies in the writing of this thesis.
Mainly it has been used for three tasks:
(1) Asking clarification questions about technical details (e.g. the role of the SW-term in the Wilson Dirac operator);
(2) Suggestion of reformulations of draft paragraphs. 
I never copied such rewritings, but eventually used them as basis for refactoring pieces of text.
Although there might be very few sentences in this document which are direct copies;
(3) Help ensure completeness when reviewing known technologies or algorithms (e.g. a list of variance reduction methods).
Responses were used as a basis.

%or similar\footnote{I even asked the AI how to cite generative AI tools properly in a PhD thesis which is a little ironic. It said one sentence is enough.}

\endgroup

