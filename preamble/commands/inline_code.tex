
% Inline code
% see https://tex.stackexchange.com/a/568900
\usepackage[most]{tcolorbox}

\tcbset{
  codebase/.style={
    enhanced,
    %fontupper=\vphantom{Ägpyj}\fontfamily{qcr}\selectfont,
    %fontupper=\vphantom{Ägpyj}\ttfamily,
    fontupper=\vphantom{Ägpyj}\fontfamily{pcr}\selectfont,
    colframe=white,
    frame style={opacity=0},
    colback=codebgcolor!12,
    nobeforeafter,
    tcbox raise base,
    shrink tight,
    extrude by=2pt,
  }
}

\newtcbox{\codebox}{codebase,      equal height group=A}
\newtcbox{\eqcodebox}{codebase,    equal height group=B}
\newtcbox{\fncodebox}{codebase,    equal height group=C}
\newtcbox{\codebreakbox}{codebase, equal height group=D}
\newtcbox{\capcodebox}{codebase,   equal height group=E}

%\def\code#1{\codebox{\texttt{#1}}}
%\def\code#1{\codebox{\mintinline[fontsize=\small]{text}{#1}}} % creates a intermediate file for EVERY \code command

\def\code#1{\codebox{#1}} % regular inline code in main text, original
\def\eqcode#1{\eqcodebox{#1}} % original
\def\fncode#1{\fncodebox{#1}} % footnote code
\def\capcode#1{\capcodebox{#1}} % caption code

% for inline \code{blabla} pieces that extend over the line break. This to manually break the line. Replace \code{bla} with \codebreak{bla} in those cases
\def\codebreak#1{\linebreak \codebreakbox{#1}}

% \usepackage{soul}
% \def\code#1{\hl{\texttt{#1}}}
% \def\codebreak#1{\hl{\texttt{#1}}}
