\chapter{QUDA}
\label{ch:p1:quda}

\readit{1}

%\worktodo{what is quda, its programming paradigms (cpp, OOP, functors, templating, abstraction, ...), its idiosyncrasies, capability of different precisions, gauge field reconstructs (gauge grp), ...}

%\worktodo{initQuda, endQuda, loadCloverQuda loadGaugeQuda newMultigridQuda, createCloverQuda, updateMultigridQuda, freeCloverQuda, destroyMultigridQuda}

In this chapter, we outline some of the main features, design choices and peculiarities of \quda, the library we offload Dirac equation solves to.
We go through an example of a simple kernel and how it is being called internally (\cref{sec:quda:kernel:invocation}), followed by how lattice fields are abstracted (\cref{sec:quda:fields}).
\Quda comes with mixed precision solvers allowing fields to change their floating-point precision (\cref{sec:quda:precision}) and it supports compressed gauge field formats exploiting the gauge group symmetry (\cref{quda:compresed:formats}).
We finish with a list of important functions exposed via \quda's header file (\cref{sec:quda:interface}) and a brief summary (\cref{sec:quda:summary}).

\tldr{what is quda}
\Quda \cite{QUDApaper} is a library written in CUDA C++ that contains a suite of efficient kernels and solvers to implement lattice QCD simulations. It can be used as a plug and play substitute for certain compute intensive tasks.
\Quda is highly optimized to run on multiple GPUs and has backends for CUDA, HIP, SYCL and multi-threading, where it performs autotuning to determine empirically the best kernel launch parameters for every kernel separately.
It thus runs efficiently on a variety of different hardware vendors.

\tldr{qudas feature set}
It has a very powerful and customizable solver suite, including a state-of-the-art multigrid implementation and comes with support for many different lattice discretizations, such as Wilson, Wilson-Clover, Twisted mass, Twisted-Clover, staggered, improved staggered, Domain-Wall and Möbius. In the meantime there is a large community of lattice practitioners actively working on and maintaining the software project.

\tldr{current release}
The current release version 1.1.0 complies with the C++14 and 17 standards and is released under an NVIDIA license similar to the MIT license.

\section{Kernel invocation}
\label{sec:quda:kernel:invocation}

\tldr{overloading, templates, functors, backend abstraction}
The codebase heavily relies on C++ features such as operator overloading and templates. Device kernels, launched on backends such as CUDA, HIP, SYCL, or threading models, are typically represented as functors -- functions that carry internal state, implemented in C++ as classes overloading the function-call operator, \code{operator()}. This approach enables seamless abstraction of implementation details from the underlying backends.
%Kernels implemented as functors are

\tldr{example of gamma5}
As an introductory example, we provide the call stack of the function \code{gamma5} in \cref{fig:quda:gamma5}.
The function applies the $\gamma^5$ matrix to an input spinor and writes the result to an output spinor.
The function itself is just a wrapper around \code{ApplyGamma} which itself calls the instantiation of the functor \code{GammaApply}.
\code{GammaApply} calls its \code{apply} method in its constructor triggering tuning for the \code{Gamma} functor.
Finally in the tuning kernels, the \code{launch\_device} call dispatches to the correct backend and instructs the runtime API to launch the kernel, for instance via \code{cudaLaunchKernel} in case of CUDA.
This executes the previously packed device function which ultimately instantiates the \code{Gamma} functor and then calls it via its function-call operator.
\begin{figure}
\includestandalone[width=\linewidth]{\dir/img/gamma5}
\caption{
Call graph of the \capcodeA{gamma5} function in \quda as an example of how kernels are abstracted from the backends.
This function acts on fields already transferred to \quda, thus there is no gamma basis transformation.
}
\label{fig:quda:gamma5}
\end{figure}
%The codebase heavily utilizes the template engine of C++.

\section{Field representations}
\label{sec:quda:fields}

\tldr{field are object instances, inheritance}
Naturally, fields on the lattice are represented as objects, and pointers to the fields' values as well as various meta information about the field are accessible via their properties and accessors.
A spinor field is represented by an instance of \code{ColorSpinorField}, a gauge field by \code{GaugeField} and a clover field as \code{CloverField}, all of which inherit from the \code{LatticeField} class. Different internal degrees of freedom are realized in the inheriting classes. The actual numbers are accessible via pointers whose accessors are called \code{data()}.

\subsection{Field transfer}

\tldr{copy/move overloading}
Copying and moving fields is efficiently handled by overloading the copy-assignment and move-assignment operators, \code{operator=}, of classes representing fields.
Fields carry meta information about whether their data lies on the CPU or GPU.
Field transfers between CPU and GPU, as well as reordering of its data, is fully abstracted away and pushed into the function bodies of the copy and move overload implementations.

%A \code{src} a \code{...} b \code{CopyColorSpinor\_} c \code{dst} bla.

\tldr{spinor equating example}
We illustrate this in \cref{fig:quda:copy}.
For example, when equating two fields of type \code{ColorSpinorField}, the copy-assignment operator is triggered which initiates the copy process.
After some instantiation of template parameters such as the floating-point precision of the source \code{src} and destination \code{dst} fields (both \code{double} in this example), spin and color degrees of freedom (\num{4} and \num{3}, respectively) and the order classes for the source and destination (more on reordering in \cref{sec:interface:field_reordering}), the copy functor \code{CopyColorSpinor\_} is launched through the tuning procedure with the correct gamma basis transformation as argument.
The remainder of the diagram after the last point (indicated by \code{...}) is very similar to the call graph of \code{gamma5}, \cref{fig:quda:gamma5}.
\begin{figure}
  \includestandalone[width=\linewidth]{\dir/img/copy}
  \caption{Copy process triggered by equating two fields of type \capcodeA{ColorSpinorField}.}
  \label{fig:quda:copy}
\end{figure}

\tldr{quda translates gamma bases}
Thus, input fields might have different $\gamma$-basis conventions than output fields.
\Quda transforms every field coming from an external application on the CPU into its preferred internal $\gamma$-basis representation and transforms back when storing the field to the applications memory region.
This is part of the data reordering procedure.

\section{Floating-point precision}
\label{sec:quda:precision}

\tldr{template make it easy for precisions}
The usage of C++ templates enables simple replacements for the precision of fields on the lattice in terms of floating-point numbers as illustrated in \cref{fig:quda:gamma5,fig:quda:copy}. \Quda allows kernels and fields to operate with the usual IEEE 754 floats like double, single, half and quarter. Assigning fields of different precision to each other triggers conversion. This makes implementations of mixed precision solvers straightforward and readable.

\section{Compression formats}
\label{quda:compresed:formats}

\tldr{internal dof with symmetries}
For fields whose internal degrees of freedom carry additional symmetry, such as the \ggrp{SU}{3}-valued gauge field, \quda provides compression formats for efficient representation in memory.
This makes sense in memory-bound scenarios, because in kernels using such a field the produced memory traffic is decreased and the arithmetic intensity increased when the field is represented in a compressed format.
For certain gauge groups \quda allows the gauge field to be stored in the following formats
\begin{itemize}
  %\item \code{QUDA\_RECONSTRUCT\_NO}: Store all \num{9} complex (\num{18} real) numbers per gauge link explicitly. This reconstruction type allows to use any arbitrary gauge group.
  \item \code{QUDA\_RECONSTRUCT\_NO}: This reconstruction type allows to use any arbitrary gauge group by storing all \num{9} complex (\num{18} real) numbers per gauge link explicitly.
  \item \code{QUDA\_RECONSTRUCT\_8}: Reconstruct each gauge link from \num{8} real numbers. This is inspired by the Lie-algebra of \ggrp{SU}{3} which has \num{8} real degrees of freedom. It is the smallest representation in terms of bytes.
  \item \code{QUDA\_RECONSTRUCT\_9}: If the \ggrp{SU}{3} matrix additionally carries a \ggrp{U}{1} phase, the Lie-algebra will have \num{9} real degrees of freedom.
  \item \code{QUDA\_RECONSTRUCT\_12}: A reconstruction type using \num{12} real numbers that is meant to be used for \ggrp{SU}{3}-valued gauge links and is more stable in reconstructing the $3 \times 3$ complex matrix.
  \item \code{QUDA\_RECONSTRUCT\_13}: Similar to the previous entry but with an additional \ggrp{U}{1} phase.
  \item \code{QUDA\_RECONSTRUCT\_10}: A \num{10}-number parameterization used for storing the momentum field.
\end{itemize}
Comparing \code{QUDA\_RECONSTRUCT\_NO} with \code{QUDA\_RECONSTRUCT\_8}, we see that the former stores \num{18} real numbers, whereas the latter only needs \num{8}. Memory boundedness of a kernel involving such a field implies roughly a speedup of $18/8 > 2$ when using the compressed format over the uncompressed one.

% making use of the modern C++ standards, an  with  making use of the object paradigm and a high degree of abstraction.
% \todo{exact standard?: by checking the CMakeFiles, QUDA v.1.0 got C++11 or 14, in v.1.1.0 needs C++14 or 17 (ref: \code{https://github.com/search?q=repo\%3Alattice\%2Fquda+c\%2B\%2B17\&type=code}, and 
% }, 

\section{Important interface functions}
\label{sec:quda:interface}

\tldr{some interface functions in quda.h}
In the following, we will list interface functions exposed by \quda's interface via \code{quda.h} that are relevant in the remainder of this document, especially for the discussion of the interface, \cref{ch:p1:interface}.
More detailed descriptions can be found in the doxygen string of each function.
These functions may trigger field transfers or reordering:

\begin{itemize}
  \item \code{initQuda()}: initialize \quda.
  \item \code{endQuda()}: finalize \quda.
  \item \code{loadGaugeQuda()}: load the gauge field from the CPU.
  \item \code{saveGaugeQuda()}: save the gauge field to the CPU.
  \item \code{freeGaugeQuda()}: free the resident gauge field in \quda.
  \item \code{loadCloverQuda()}: load the clover field from the CPU.
  \item \code{freeCloverQuda()}: free the resident clover field in \quda.
  \item \code{newMultigridQuda()}: create a new multigrid instance.
  \item \code{updateMultigridQuda()}: update an existing multigrid instance.
  \item \code{destroyMultigridQuda()}: destroy an existing multigrid instance.
  \item \code{invertQuda()}: call the inverter on CPU fields.
  \item \code{eigensolveQuda()}: call the eigensolver on CPU fields.
  \item \code{MatQuda()}: call the Dirac operator on CPU fields.
  \item \code{plaqQuda()}: calculate the plaquette over the resident $\ggrp{SU}{3}$ gauge field.
\end{itemize} 

\section{Summary}
\label{sec:quda:summary}

%\worktodo{summary of quda}

\Quda is a lattice QCD library running efficiently on GPUs of various vendors via CUDA, HIP and SYCL backend abstractions.
It comes with a state-of-the-art mixed-precision multigrid solver optimized for many lattice QCD Dirac operators such as the Wilson-Clover one.
Abstraction is practiced in terms of operator overloading, class inheritance and heavy templating.
It offers a basic interface for usage as a library.
\Cstar boundary conditions and the QCD+QED Wilson-Clover Dirac operator are missing features which we implemented.
