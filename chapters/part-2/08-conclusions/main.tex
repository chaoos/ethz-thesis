\chapter{Conclusions}
\label{ch:p2:conclusions}

\worktodo{
* Conclusions
%	* Staggered
	* Outlook
%		* other observables, 1pt function, baryon corrs, ...
%	* most lattice discretizations differ in their treatment of chirality making MG a challange to be studied. It has to be studied separately on all of them
	* reaching gauge noise is hard. multi-level schemes (that reduce gauge noise) make it even harder
}

We have introduced a novel variance reduction method called multigrid low-mode averaging, combining the efficient multigrid preconditioner with low-mode deflation of correlators.
The method can be applied to any $n$-point function, but benefits are only expected if the observable is sensitive to low-mode noise.
This is the case for a large class of observables including meson two-point functions important in the high-precision determination of the leading-order HVP of the muon $g-2$, but also baryon three-point functions important for hadron spectroscopy.
For disconnected diagrams leading to one-point functions, we do not expect a huge gain in variance reduction, since the success of methods like frequency splitting suggests that the noise comes from the high modes, rather than the low modes.
However, as a rule of thumb, whenever LMA is beneficial, MG-LMA should be too.

The insights when studying chirality on the subspaces and the fact that most lattice discretizations differ in their treatment of chiral symmetry, leads to the conclusion that MG-LMA has to be investigated and possibly adjusted on each discretization separately.
Although, the introduced method will constantly benefit from the rapid theoretic and algorithmic developments of multigrid algorithms on different lattice discretizations.

\tldr{twisted mass and MG}
For twisted mass fermions multigrid had to be adjusted by introducing a pure algorithmic parameter -- a factor for the coarse twisted mass~\cite{Alexandrou:2016izb} -- to improve the condition of the coarse grid twisted mass Dirac operator.
If not adjusted, the solver performance might degrade or stall entirely.
However, the coarse operator is not the fine grid restriction anymore if that factor is unequal to one.
This trade off might have impact into the variance contribution and might degrade the performance of MG-LMA compared to Wilson fermions.

\tldr{staggered and MG}
The staggered fermion Dirac operator shows spurious eigenvalues on coarse grids~\cite{Brower:2018ymy} making it especially challenging to obtain a performant multigrid implementation.
The solution seems to be a transformation of the operator by the Kähler-Dirac spin structure~\cite{Becher:1982ud,Bodwin:1987ah} before coarsening.
This results in a spectrum similar to the Wilson Dirac one and multigrid can be successfully formulated on the Kähler-Dirac transformed operator which was verified on the Schwinger model~\cite{Brower:2018ymy} and later on full lattice QCD~\cite{Ayyar:2022krp}.
A straightforward application of MG-LMA without taking these developments into consideration is thus not expected.

\tldr{domain wall and MG}
For Domain Wall fermions with its Dirac operator spectrum enclosing the origin one usually coarsens the normal HPD system $D^{\dagger} D$~\cite{Cohen:2011ivh,Boyle:2014rwa} or the Pauli-Villars operator~\cite{Brower:2020xmc}.
Although the second option has only been tested on the 2D Schwinger model.
As proven in \cref{thm:cond:hpd}, for the HPD system, the coarse operators will be at least as good conditioned as the fine one, no matter if chirality is preserved or not.
Admittedly this operator is more expensive to apply and complex to coarsen as it includes next-to-nearest neighbor couplings.
On the other hand, when naively coarsening $D$, coarse operators show spurious eigenvalues~\cite{Brower:2020xmc} similar to the staggered case.
Again, we do not expect a naive application of MG-LMA to the Domain Wall discretization to be successful without considering the developments made in recent multigrid literature.

\tldr{concluding discretizations}
Multigrid low-mode averaging -- despite being very successful on the Wilson discretization as shown -- has to be studied for all of the above discretizations individually and carefully.
There is still a lot of theoretic and algorithmic work to do.

\tldr{outlook: baryons}
A promising imminent application without fundamental changes would be three-point functions which emerge when considering baryon correlators.
They manifest a severe signal-to-noise ratio problem and low-mode averaging is applied to these problems with success.
We expect a clear improvement of MG-LMA when applied.
%Most promising might be an application to three-point functions which emerge when considering baryon correlators.


