
\newcommand{\ie}{i.\,e.\xspace}
\newcommand{\Ie}{I.\,e.\xspace}
\newcommand{\eg}{e.\,g.\xspace}
\newcommand{\Eg}{E.\,g.\xspace}

\newcommand{\quda}{QUDA\xspace}
\newcommand{\Quda}{QUDA\xspace}
\newcommand{\qudas}{QUDAs\xspace}
\newcommand{\Qudas}{QUDAs\xspace}
\newcommand{\openqxd}{openQxD\xspace}
\newcommand{\openqcd}{openQCD\xspace}
\newcommand{\Openqxd}{OpenQxD\xspace}
\newcommand{\Openqcd}{OpenQCD\xspace}
\newcommand{\bigO}{\mathcal{O}}
\newcommand{\observable}[1]{\mathcal{#1}}
\newcommand{\NULL}{NULL\xspace}
\newcommand{\posix}{posix\xspace}
\newcommand{\Posix}{Posix\xspace}

\newcommand{\txyz}{txyz\xspace}
\newcommand{\xyzt}{xyzt\xspace}

\newcommand{\QED}[1]{QED$_\text{#1}$\xspace}
\newcommand{\QEDC}{\QED{C}}

\newcommand{\lat}[1]{\Lambda_{\text{#1}}}
\newcommand{\RCstar}{RC$^{\star}$\xspace}
\newcommand{\Cstar}{C$^{\star}$\xspace}
\newcommand{\CstarHeading}{C\texorpdfstring{$^{\star}$}{*}\xspace}

\newcommand{\x}{×\xspace}

\newcommand{\ggrp}[2]{#1(#2)}
\newcommand{\liealg}[2]{\mathfrak{#1}(#2)}

\newcommand{\cswsu}[1]{c_{\text{SW}}^{\ggrp{SU}{#1}}}
\newcommand{\cswu}[1]{c_{\text{SW}}^{\ggrp{U}{#1}}}

\newcommand{\Dw}{{D_{\mathrm{w}}}}

\newcommand{\openmp}{openMP\xspace}
\newcommand{\Openmp}{OpenMP\xspace}

\newcommand{\openmpi}{openMPI\xspace}
\newcommand{\Openmpi}{OpenMPI\xspace}

\newcommand*\dd{\mathop{}\!\mathrm{d}}
\newcommand*\DD{\mathop{}\!\mathcal{D}}

\newcommand{\spinhalf}{spin-$\frac{1}{2}$\xspace}
\newcommand{\Nst}{N_{\text{st}}} % number of stochastic sources
\newcommand{\Nc}{N_{\text{c}}} % number of eigenmodes sources
\newcommand{\Ns}{N_{\text{s}}} % number of spin dof
\newcommand{\id}{\mathbb{1}} % identity matrix

\newcommand{\Lt}{L_0} % extent in time
\newcommand{\Lx}{L_1} % extent in x-direction
\newcommand{\Ly}{L_2} % extent in y-direction
\newcommand{\Lz}{L_3} % extent in z-direction

\newcommand{\vect}[1]{\vec{#1}}
%\newcommand{\vect}[1]{\mathbf{#1}}

\newcommand{\area}[1]{\text{Area}(#1)}

\newcommand{\exptime}{\left\{ 0, \ldots \Lt-1 \right\}}  % explicit temportal lattice set
\newcommand{\expspin}{\left\{ 0, \ldots N_s-1 \right\}}  % explicit spin index set
\newcommand{\expcolor}{\left\{ 0, \ldots \Nc-1 \right\}} % explicit color index set
\newcommand{\expspace}{\left\{ \vect{x} = (x_1,x_2,x_3) \in \mathbb{N}^3 \; \middle| \; 0 \leq x_i < L_i \right\}}

%\newcommand{\ispin}{\text{spin}} % spin index set
%\newcommand{\icol}{\text{color}} % color index set
%\newcommand{\ilat}{\text{lattice}} % lattice index set
%\newcommand{\ispace}{\text{space}} % spatial lattice index set
\newcommand{\stime}{\{ 0, \ldots \Lt-1 \}} % lattice time set explicit

\newcommand\vslattice{\mathcal{V}}  % lattice vector space

% notation for coarse objects; hat or widehat
\usepackage{xparse} % loads expl3
\ExplSyntaxOn
\NewDocumentCommand{\coarse}{m}{ % coarse operator
    \int_compare:nTF { \tl_count:n { #1 } > 1 }
        { \widehat{#1} }
        { \hat{#1} }
}
\ExplSyntaxOff

\newcommand{\nrad}[1]{\omega( #1 )}
\newcommand{\inrad}[1]{\omega_{\text{min}}( #1 )}

\newcommand{\restrictor}{R}       % restrictor
\newcommand{\prolongator}{T}      % prolongator

\newcommand{\tslice}[1]{\langle {#1} \rangle} % time slice notation for stochastic estimator

\newcommand{\lattice}{\mathbb{L}}       % represents the lattice
\newcommand{\idxset}{\mathcal{I}}       % all lattice indices
\newcommand{\idxspace}{\Sigma}          % spatial lattice set
\newcommand{\idxtime}{\mathcal{T}}      % temporal lattice set
\newcommand{\idxspacetime}{\Lambda}     % spacetime lattice set
\newcommand{\idxspin}{\Xi}              % spin index set
\newcommand{\idxcolor}{\mathfrak{C}}    % color index set
\newcommand{\idxchiral}{\mathfrak{X}}   % chirality index set

\newcommand{\cexpcolor}{\left\{ 0, \ldots \coarse{\Nc}-1 \right\}}

% spinor product, scalar product
\newcommand{\sprod}[2]{\left({#1}, {#2}\right)} % (A, B)
%\newcommand{\sprod}[2]{\langle #1, #2 \rangle)} % <A, B>
%\usepackage{braket}
%\newcommand{\sprod}[2]{\braket{#1}{#2}} % <A | B>

\newcommand{\evec}{\xi} % symbol for the eigen spinors


% propagator
\newcommand{\prop}{S}
%\newcommand{\prop}{D^{-1}}


\newcommand{\opxy}[3]{{#1}_{[{#2} ; {#3}]}} % Op_{[x;y]}
\newcommand{\propxy}[2]{\opxy{\prop}{#1}{#2}} % S_{[x;y]}
%\newcommand{\propxyab}[6]{{\prop_{[{#1};{#2}]}}_{[{#3} {#4}] \atop [{#5} {#6}]}}
\newcommand{\propxyab}[6]{\prop_{\begin{smallmatrix}
[{#1} ; {#2}] & [{#3} {#4}] \\
              & [{#5} {#6}]
\end{smallmatrix}}}
\newcommand{\fieldx}[2]{{#1}[{#2}]}
\newcommand{\fieldxa}[3]{{{#1}[{#2}]}_{{#3}}}
\newcommand{\fieldxaa}[4]{{{#1}[{#2}]}_{\begin{smallmatrix}
[{#3}] \\
[{#4}]
\end{smallmatrix}}}
\newcommand{\opxyab}[7]{{#1}_{\begin{smallmatrix}
[{#2} ; {#3}] & [{#4} {#5}] \\
              & [{#6} {#7}]
\end{smallmatrix}}}

\newcommand{\vspan}[1]{\text{span} \left\{ {#1} \right\}}

\newcommand\subsetsim{\mathrel{%
  \ooalign{\raise0.2ex\hbox{$\subset$}\cr\hidewidth\raise-0.8ex\hbox{\scalebox{0.9}{$\sim$}}\hidewidth\cr}}}

\newcommand{\gramschmidt}[1]{\mathcal{G}\mathcal{S}\left( {#1} \right)}

\newcommand{\Nlvl}{N_{\text{lvl}}} % number of multigrid levels
\newcommand{\lvl}{\ell} % level l
\newcommand{\onlvl}[2]{{#1}^{({#2})}}
%\newcommand{\onlvl}[2]{{#1}_{L_{#2}}}
\newcommand{\Ln}[1]{\text{$\mathrm L #1$}}

% cf. -> compare
\newcommand{\cf}{cf.\xspace}

\newcommand{\Nf}{N_f} % number of flavors
\newcommand{\Nconf}{N} % number of configs
\newcommand{\Niter}{N_{\text{it}}} % number of configs
\newcommand{\Nrhs}{N_{\text{rhs}}} % number of right hand sides

\newcommand{\ratio}{\Phi} % ratio of fine / coarse inversion
\newcommand{\cost}{\text{cost}} % symbol for cost
\newcommand{\iter}{\text{iter}} % symbol for iteration count
\newcommand{\mem}{\text{mem}} % symbol for memory bandwidth
\newcommand{\speedup}{\text{Sp}} % symbol for speedup
\newcommand{\sizeof}{\text{sizeof}}


% chirality stuff

\newcommand{\convhull}[1]{\text{conv}\{ #1 \}}
\newcommand{\convshull}[1]{ \convhull{ \sigma( #1 ) } }

\newcommand{\csym}{\text{ID}} % Symbol for cases
\newcommand{\csyms}{\text{IDs}} % Symbols for cases
\newcommand{\boxcheck}{
    \makebox[0pt][l]{$\square$}\raisebox{.15ex}{\hspace{0.1em}$\checkmark$}
}

%\newcommand{\case}[1]{$\csym = #1$}
\newcommand{\caseX}[1]{(\hyperlink{id:#1}{#1})}
\newcommand{\pcaseX}[1]{(\protect\hyperlink{id:#1}{#1})}
\newcommand{\caseXtwo}[2]{(\hyperlink{id:#1}{#1}) and (\hyperlink{id:#2}{#2})}
\newcommand{\caseXthree}[3]{(\hyperlink{id:#1}{#1}), (\hyperlink{id:#2}{#2}) and (\hyperlink{id:#3}{#3})}
\newcommand{\caseXfour}[4]{(\hyperlink{id:#1}{#1}), (\hyperlink{id:#2}{#2}), (\hyperlink{id:#3}{#3}) and (\hyperlink{id:#4}{#4})}
\newcommand{\rowtarget}[2]{\hypertarget{id:#1}{#2}}