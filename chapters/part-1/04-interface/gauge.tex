%\documentclass{scrartcl}
\documentclass{standalone}

%\pagestyle{empty}

\usepackage{tikz,ifthen}

\usetikzlibrary{arrows.meta}
\usetikzlibrary{calc}
\usetikzlibrary{decorations.markings}
\usetikzlibrary{decorations}
\usetikzlibrary{positioning}

%
% Customize colors
%
\definecolor{chapter-color}{cmyk}{1, 0.50, 0, 0.25}
\definecolor{link-color}{cmyk}{1, 0.50, 0, 0.25}
\definecolor{cite-color}{cmyk}{0, 0.7, 0.9, 0.2}
\definecolor{codegreen}{rgb}{0,0.6,0}
\definecolor{codegray}{rgb}{0.5,0.5,0.5}
\definecolor{codepurple}{rgb}{0.58,0,0.82}
\definecolor{backcolour}{rgb}{0.95,0.95,0.92}
\definecolor{codebgcolor}{RGB}{129, 139, 152}
\definecolor{codehighlightcolor}{RGB}{255, 230, 153}
%\definecolor{codegreen}{RGB}{0, 153, 0}
%\definecolor{codegray}{RGB}{127, 127, 127}
\definecolor{codeblue}{RGB}{102, 214, 237}
\definecolor{codekeyword}{RGB}{249, 36, 114}
\definecolor{codecomment}{RGB}{127, 127, 127}
\definecolor{backcolor}{RGB}{242, 242, 235}
\definecolor{linkcolor}{RGB}{102, 0, 0}
\definecolor{corange}{RGB}{255, 70, 0}
\definecolor{cyellow}{RGB}{209, 153, 0}
\definecolor{cblue}{RGB}{64, 128, 255}
\definecolor{cbrown}{RGB}{153, 102, 51}
\definecolor{cpink}{RGB}{255, 0, 255}
\definecolor{cred}{RGB}{255, 64, 0}
\definecolor{cgreen}{RGB}{0, 191, 0}
\definecolor{clightblue}{RGB}{191, 217, 255}
\definecolor{cturquois}{RGB}{0, 255, 255}
\definecolor{cpurple}{RGB}{128, 0, 255}
\definecolor{clightgreen}{RGB}{175, 255, 175}
\definecolor{clightgray}{RGB}{211, 211, 211}
\definecolor{clightpink}{RGB}{255, 175, 255}
\definecolor{cdarkblue}{RGB}{0, 0, 255}
\definecolor{cdarkred}{RGB}{255, 0, 0}
\definecolor{cdarkgreen}{RGB}{0, 255, 0}
\definecolor{cgray}{RGB}{153, 153, 153}

\definecolor{myblue}{RGB}{55, 126, 184}
\definecolor{myorange}{RGB}{255, 127, 0}
\definecolor{myred}{RGB}{228, 26, 28}
\definecolor{mypurple}{RGB}{152, 78, 163}
\definecolor{mygreen}{RGB}{77, 175, 74}
\definecolor{myyellow}{RGB}{255, 255, 51}
\definecolor{mybrown}{RGB}{166, 86, 40}
\definecolor{mypink}{RGB}{166, 86, 40}
\definecolor{mygray}{RGB}{153, 153, 153}


\begin{document}

\begin{tikzpicture}

% openQxD
\foreach \x in {0,1,2,3} {
    \foreach \y in {0,1,2,3} {
        \coordinate (n) at ($\x*(20mm, 0)+\y*(0, 20mm)$);
        \coordinate (A1) at ($\x*(20mm, 0)+\y*(0, 20mm)+(0,15mm)$);
        \coordinate (B1) at ($\x*(20mm, 0)+\y*(0, 20mm)+(0,5mm)$);
        \coordinate (A2) at ($\x*(20mm, 0)+\y*(0, 20mm)+(15mm,0)$);
        \coordinate (B2) at ($\x*(20mm, 0)+\y*(0, 20mm)+(5mm,0)$);
        \coordinate (A3) at ($\x*(20mm, 0)+\y*(0, 20mm)-(0,15mm)$);
        \coordinate (B3) at ($\x*(20mm, 0)+\y*(0, 20mm)-(0,5mm)$);
        \coordinate (A4) at ($\x*(20mm, 0)+\y*(0, 20mm)-(15mm,0)$);
        \coordinate (B4) at ($\x*(20mm, 0)+\y*(0, 20mm)-(5mm,0)$);
        \ifthenelse{\x=0 \and \y=0} {
            \filldraw[color=cred,fill=cred](n) circle (0.1); % origin point
        } {
            \pgfmathparse{int(mod(\x+\y,2))};
            \ifnum\pgfmathresult=0 {
                \filldraw[color=black,fill=black](n) circle (0.1); % even points
            } \else {
                \draw[color=black,thin](n) circle (0.1); % odd points
                \draw[->] (A1) -- (B1);
                \draw[->] (A2) -- (B2);
                \draw[->] (A3) -- (B3);
                \draw[->] (A4) -- (B4);
            } \fi
        };
    };
};

% QUDA
\foreach \x in {0,1,2,3} {
    \foreach \y in {0,1,2,3} {
        \coordinate (n) at ($\x*(20mm, 0)+\y*(0, 20mm) + (100mm,0)$);
        \coordinate (A1) at ($\x*(20mm, 0)+\y*(0, 20mm)+(0,15mm) + (100mm,0)$);
        \coordinate (B1) at ($\x*(20mm, 0)+\y*(0, 20mm)+(0,5mm) + (100mm,0)$);
        \coordinate (A2) at ($\x*(20mm, 0)+\y*(0, 20mm)+(15mm,0) + (100mm,0)$);
        \coordinate (B2) at ($\x*(20mm, 0)+\y*(0, 20mm)+(5mm,0) + (100mm,0)$);
        \ifthenelse{\x=0 \and \y=0} {
            \filldraw[color=cred,fill=cred](n) circle (0.1); % origin point
        } {
            \pgfmathparse{int(mod(\x+\y,2))};
            \ifnum\pgfmathresult=0 {
                \filldraw[color=black,fill=black](n) circle (0.1); % even points
            } \else {
                \draw[color=black,thin](n) circle (0.1); % odd points
            } \fi
        };
        \draw[->] (A1) -- (B1);
        \draw[->] (A2) -- (B2);
    };
};
\end{tikzpicture}

\end{document}