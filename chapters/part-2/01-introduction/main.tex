\chapter{Introduction to Part II}
\label{ch:p2:introduction}

\dictum[Nazario Tantalo]{%
  I used to say that discussions about noise are simply noise ...}%
\vskip 1em

\readit{2}

% \worktodo{
% a) physics: we need precise correlators -> little recap on what I already said
% 	* motivation for variance reduction methods
% 	* g-2 HVP and so on (short)
% 	* total: 1-2p maximal
% b) translation averaging suppresses the gauge noise sim 1/sqrtV but exact translation averaging costs V2 -> estimate
% 	* definintion of translation averaging on an n-point function -> see our PRD
% 	* follow this intro: https://arxiv.org/pdf/2401.14724
% 	* use correlator expression from intro -> g-2 > HVP -> final few eqs
% 	* total 1-3p
% c) stochastic estimators (Hutchinson) add extra variance which can be large
% 	* we want translation average, but it's V^2 with point sources -> estimate
% 	* standard hutchinson 1p
% 	* maybe dilution here 1p
% d) existing solutions: probing/freq. splitting/LMA => motivation for next chapter
% 	* connected
% 		* LMA 4p
% 	* disconnected
% 		* HP 1p
% 		* FS 1p
% 		* MLMC 1p
% 		* MG MLMC 1p
% e) other problems/related literature (distillation, Parisi-Lepage StN, ...) what this thesis doesn't cover but you want to acknowledge
% 	* StN, 2p
% 	* multilevel sampling 2p
% 	* (smearing 1p) (evtl. remove details, but leave the refs)
% 	* (distillation 1p) (evtl. remove details, but leave the refs)
% 	* (end smearing 1p) (evtl. remove details, but leave the refs)
% 	* dilution 1p (evtl. after hutchinson)
% f) remainder of this part
% 	* last paragraph 1p
% 	* (Add LMA chapter)
% }

%\worktodo{a)}
\tldr{HVP needs variance reduction methods}
Precise observable evaluations in the context of numerical lattice field theory require significant computational resources.
Requirements in error budget pose challenges, both for systematic and statistic error estimations.
This opens the vast research field of variance reduction methods.
As outlined in \cref{sec:gm2:contribution}, the dominant uncertainty of the current SM prediction of the muon's anomalous magnetic moment arises from the leading-order HVP contribution~\cite{snowmass:2025}.
Since the minimal (gauge) variance is usually not reached, powerful variance reduction methods are needed to tackle the light-quark connected part in particular, which suffers from infeasible translation average cost at large distances.

%\worktodo{b)}
\tldr{translation invariance}
Since the QCD action and the path integral measure are invariant under spacetime translations, $n$-point functions such as \cref{eq:C_corr_xy} are translation invariant too; for all spacetime points $y$ it holds
\begin{equation} \label{eq:translation:average}
C_{\mu \nu}(x_1; x_2) = C_{\mu \nu}(x_1 + y; x_2 + y) \;.
\end{equation}
For quark-line connected correlation functions on the lattice, the variance is expected to decrease inversely proportional to the averaged volume due to the mass gap in QCD~\cite{Luscher:2017cjh}.
This means that the Euclidean time correlator \cref{eq:coor1} can be averaged over all source lattice points $y$ resulting in minimal variance -- variance related only to the fluctuations of the background gauge field.
This minimal variance which we will call the gauge variance will be defined precisely and determined in \cref{sec:numerics:gauge:variance}.
It gives us a lower bound of what is possible to achieve, but also a variance reduction goal for given gauge field configurations.

\tldr{naive evaluation is V2}
Naively evaluating \cref{eq:coor1} for a single source point $y$ has an associated cost of \num{12} Dirac equation solves as outlined in \cref{ch:p2:2pt-corr}.
Thus, a full volume average scaling as $\bigO(V^{2})$ is infeasible.
Although, we can calculate an estimation of the volume averaged quantity using improved estimators, for instance the standard method introduced in the following paragraph.

%Looking at the observable \cref{eq:coor1}, 

%\worktodo{c)}
\tldr{simple Hutchinson}
A standard estimator for volume averages of traces over propagators is the Hutchinson method~\cite{Hutchinson01011990}, where a stochastic identity is introduced for an estimation of single inverse matrix traces.
As a negative byproduct, this introduces additional stochastic noise on top of the gauge noise which has to be suppressed separately.
However, this enables stochastic all-to-all propagators~\cite{Foley:2005ac} when using random wall-sources; sources with support only on a certain time slice.
A necessity stemming from correlation functions often being investigated as functions of Euclidean time.
This introduces an implicit spatial lattice volume average as discussed in the previous paragraph and is expected to reduce the variance proportional to the spatial volume.
The stochastic noise can be any set with zero mean and unit variance, like Gaussian or subsets thereof; a popular choice is $\mathbb{Z}_2 = \{\pm 1\}$ which minimizes stochastic variance over traces of real matrices~\cite{Bernardson:1993he}.

\tldr{index dilution}
Further developments of stochastic trace estimation led to dilution schemes (also called partitioning in earlier works~\cite{Wilcox:1999ab}).
Random sources with support only on certain spin~\cite{Wilcox:1997rm,Alexandrou:2010jr}, color, time~\cite{GUSKEN1989,OCais:2004xgm,Bali:2014pva}, space slices~\cite{Foley:2005ac}, even-odd indices or combinations thereof~\cite{Morningstar:2008mc,Bulava:2008qx,Foley:2010vv,Morningstar:2008ph}.
Apart from positive impact on the variance, dilution also has algorithmic consequences.
Spin dilution for instance makes it possible with just \num{4} inversions to reconstruct every $\gamma$-matrix monomial possibly appearing in an interpolation operator.
This makes it possible to evaluate multiple diagrams with the same \num{4} spin-diluted noise sources.
Dilution on an internal degree of freedom like color and/or spin tends to reduce the overall stochastic noise introduced by the stochastic identity, now being exactly zero on off-diagonal dilution indices~\cite{Babich:2010at,Morningstar:2011ka}.
Variance caused by even-odd interactions can be cancelled by even-odd dilution~\cite{Bali_2009,Foley:2005ac,Morningstar:2011ka}.
The structure of the observable should always be a guidance for an effective dilution scheme.
Clearly the extreme case would be a dilution in all indices; color, spin, time and space, resulting in an exact estimation using point sources.% but at infeasible cost.

%\worktodo{d)}
\tldr{stochastic estimators reduce variance globally and are V2 too}
Standard stochastic estimators as the ones introduced in the last paragraphs reduce variance globally across all Euclidean times of a correlator $G(t)$, \ie uniformly over the lattice.
They are agnostic to the physical observable or the region in Euclidean time where variance reduction is most important.
On increasing volumes, such estimators require thousands of sources to resolve the large distance properly and a $\bigO(V^{2})$ scaling emerges again when attempting the true volume average.

\tldr{signal/noise ratio problem and parisi/lepage}
% Precise observable evaluations in the context of numerical lattice field theory require significant computational resources.
% Requirements in error budget pose challenges, both for systematic and statistic error estimations.
% This opens the vast field of research of variance reduction methods.
Many hadronic Euclidean time correlators such as the connected LO-HVP suffer a systemic phenomenon called the signal-to-noise ratio problem~\cite{parisi1984,lepage1989}.
For two-point correlation functions in the time-momentum representation~\cite{Bernecker:2011}, both the signal as well as the gauge noise decay exponentially in Euclidean time but at different rates.
%\worktodo{Tim: Here again, you are considering two-point functions (in the time-momentum representation, for instance)}
This means that the correlator signal gets lost in the gauge noise at large distances.
%It states that the noise of the correlator starts to dominate the signal in the large distance regime.
Where the signal decays as $e^{-E_0 t}$ with $E_0 > 0$ the lowest energy state, the gauge noise, which is proportional to the magnitude squared of the correlator, decays as $e^{-E_{\text{noise}}t/2}$ where $E_{\text{noise}} > 0$ is the energy of the lightest state of the squared correlator.
The ratio of the two $e^{-(E_0 - E_{\text{noise}}/2)t}$ decays exponentially.
If $2E_0 > E_{\text{noise}}$ the signal deteriorates especially for large Euclidean times $t$.

\tldr{pion corr is good, baryons, mesons bad}
The pion correlator for instance has $E_0 = m_{\pi}$ the pion mass and $E_{\text{noise}} = 2 m_{\pi}$ leading to a constant signal-to-noise ratio.
The signal of that correlator clearly does not deteriorate as time increases, since the pion has the lightest mass among the hadrons.
Following the analysis of Parisi and Lepage, $E_{\text{noise}} = 3/2 m_{\pi}$ for baryon correlators where $E_0 = m_{b}$ the baryon mass.
Since baryon masses are in the regime of one or multiple GeV and the pion masses are about \SI{135}{\mega \eV}, we observe a signal loss already early in the decay.
For the isovector light-quark connected LO-HVP correlator, we find $E_0 = m_{\rho}$ the mass of the $\rho$-meson which is about \SI{770}{\mega \eV} and $E_{\text{noise}} = 2 m_{\pi}$ as for baryons -- again a loss in signal at large distances.

\tldr{LMA treats LD specifically}
According to the last paragraphs, variance reduction should therefore systematically target the large distance part of the connected LO-HVP.
A method called low-mode averaging (LMA)~\cite{Neff_2001,DeGrand_2004,Giusti_2004}, which will be discussed in detail in \cref{ch:p2:lma}, deflates low eigenmodes of the Dirac operator and enables full spacetime lattice volume averaging over the correlator restricted to this deflated low-mode subspace.
This is justified by the fact that low-lying eigenmodes govern the infrared (long distance) behavior of the theory~\cite{DeGrand_2004}.
If enough low modes have been deflated, this results in a significant reduction of variance in the long distance.
Nearly all current high-precision determinations of the connected LO-HVP employ LMA for this reason~\cite{snowmass:2025,PhysRevLett.121.022003,bmw_2021,PhysRevD.101.074515,PhysRevD.107.034513,Aubin:2022hgm,ExtendedTwistedMass:2022jpw,PhysRevD.108.054507,bmw_2024,Spiegel:2024dec,PhysRevLett.134.201901,Djukanovic:2024cmq,milc_gm2,FermilabLatticeHPQCD:2024ppc}.

\tldr{LMA is V2 too}
To resolve the large distance of the connected LO-HVP contribution via translation averaging introduced by LMA, the number of low modes has been driven to the limit of affordability; \numrange{1000}{6000}~\cite{Djukanovic:2024cmq,RBC_2024,bmw_2024,Aubin:2022hgm} still without reaching gauge variance.
The number of low modes required for constant variance reduction is proportional to the lattice volume, because the size of the low-mode subspace is so too~\cite{banks1980}.
Therefore, the computational work to determine this subspace scales at least quadratic in the lattice volume and again the $\bigO(V^{2})$ scaling hits us, limiting the achievable precision.
%, which we will refer to as \emph{$V^{2}$-problem}~\cite{Luescher2007}.
This fact renders LMA on large-scale lattices prohibitively expensive, making it the limiting factor in statistical precision of the connected LO-HVP.

\tldr{we call it V2 problem}
This ever emerging quadratic scaling in the lattice volume, closely related to critical slowing down, we will refer to as the $V^{2}$-problem~\cite{Luescher2007}.

%\worktodo{disconnected}
\tldr{intro disconnected}
In the following, we will briefly review some related variance reduction methods targeting disconnected diagrams, since some of the motives were carried over to developments in this thesis.

\tldr{hierarchical probing}
For disconnected diagrams which are closely related to single inverse matrix traces, methods based on specific choice of probing vectors have been developed.
They go under the name of hierarchical probing~\cite{tang2012probing,Stathopoulos:2013aci}.
%, in the assumption that the off-diagonal elements of the inverse of a nearest neighbor stencil operator fall off exponentially fast.
Based on the analysis in ref.~\cite{Hutchinson01011990}, the variance of the trace is the Frobenius norm of the matrix with zeroed diagonal entries.
The probing method builds structured vectors that systematically cancel noise coming from short distance on nearby lattice sites resulting in significantly reduced variance as compared to standard stochastic estimators for disconnected diagrams~\cite{Stathopoulos:2013aci}.

	% * frequency splitting: stochastic estimation of disconnected diagrams, ...
	% 	* REF: https://arxiv.org/pdf/1903.10447
	% 	* REF: https://arxiv.org/pdf/2001.08783

\tldr{frequency splitting}
In a very similar tone, the method called frequency splitting~\cite{Giusti:2019kff,Giusti:2020dtz} separates high and low frequencies associated to short and long distances in disconnected diagrams by introducing a telescopic sum in the quark masses.
The propagator itself is split, turning the disconnected diagram into possibly multiple pieces; the short distance (high frequency) part is split into a bulk which is calculated exactly (\ie at its gauge noise level) and a remainder, estimated stochastically, that contributes little to the variance.
The remaining long distance (low frequency) part (differences of quark propagators with different masses) can then very efficiently be estimated with stochastic split-even estimators making use of the identity $D_1^{-1} - D_2^{-1} = (m_2 - m_1) D_2^{-1} D_1^{-1}$ where $D_i$ is the Dirac operator with mass $m_i$~\cite{Whyte:2022vrk}.
This is sometimes referred to as the one-end trick in context of twisted mass fermions~\cite{ETM:2008zte,PhysRevD.73.074506,PhysRevD.59.074503}.
Crucial for this method is that the Dirac operator is better conditioned for large quark masses.
%However, this exploits the fact that the Dirac operator is better conditioned for large quark masses.

	% * MGMLMC: single inverse matrix traces, discon. diags
	% 	* REF: https://inspirehep.net/literature/2612876
	% 	* REFS: Frommer:2025kfp
	%
	% 	doi:10.1137/21M1441894:2022

\tldr{Multilevel Monte Carlo}
We want to briefly discuss multilevel Monte Carlo (MLMC) methods~\cite{giles2008,Giles_2015}.
%Unfortunately there is a naming clash: this has nothing to do with multi-level algorithms for gauge field generation as discussed previously.
Multilevel Monte Carlo is a general framework to split an unbiased stochastic estimator into a sum of estimators.
The individual terms contribute differently to the overall variance and one hopes to find decompositions where terms dominating the variance are cheap to estimate.
MLMC does not propose actual decompositions but formalizes the general procedure, describing the heart of nearly every variance reduction method.

\tldr{MG-MLMC}
In this spirit, deflation of low modes was used to estimate traces over inverse matrices in~\cite{Gambhir:2016uwp,Alcalde_2017}, but speedups where only obtained together with hierarchical probing.
A method called multigrid multilevel Monte Carlo (MG-MLMC) decomposes the propagator into multigrid levels using an adaptive algebraic multigrid approach and then estimates the trace separately on all levels.
This was investigated on the 2D (gauge-covariant) Laplacian and the Schwinger model~\cite{doi:10.1137/21M1441894:2022}.
A similar method was previously been used for trace estimation of single inverse matrices using inexact deflation à la Lüscher~\cite{Romero:2019psj}.
It was recently improved to use the Hutch++ method on coarse levels~\cite{Frommer:2022qiy}.
The Hutch++ method~\cite{doi:10.1137/1.9781611976496.16:2021} decomposes the trace into a projected low-rank term computed exactly, whereas the remainder is estimated using standard Hutchinson~\cite{Hutchinson01011990}.
First tests on disconnected QCD diagrams where conducted~\cite{Frommer:2025kfp} resulting in a moderate variance reduction of about \SI{12}{\percent} compared to standard Hutchinson.
%A very similar study was already performed in ref~\cite{Alcalde_2017,} with comparable results, speedup where only obtained when combining the method with hierarchical probing.

\tldr{variance in the high modes}
This is to be expected, since the stochastic variance in disconnected diagrams stems from the high modes, not from the low modes.
The variance of an random inverse matrix trace comes from the off-diagonal parts.
In case of the quark propagator the off-diagonal parts decrease exponentially.
Therefore it is the closest off-diagonal part that contribute dominantly.
These are short distanced, \ie originate from the high modes.
Hierarchical probing reduces short-distance off-diagonal noise by introducing special probing vectors.
Frequency splitting splits off the high modes using the propagator difference $D_1^{-1} - D_2^{-1}$.
Both methods attempt to estimate this high frequency part more precisely resulting in a net-variance reduction.
Low-mode deflation as is done with MG-MLMC is thus not expected to be beneficial for such traces.
%\worktodo{Tim: There needs to be a conclusion of this section!!!}

%\worktodo{e)}
\tldr{intro unrelated methods}
What will follow are other methods only loosely related to this thesis, since they target reduction of variance on different quantities and problems, but are nevertheless worth mentioning for additional context.

\tldr{multi-level integration}
The fundamental signal-to-noise restriction on correlation functions discussed above led to the development of algorithms employing integration techniques with multiple levels.
Such multi-level algorithms\footnote{We note that multi-level Monte Carlo schemes for propagator computations is not related to the multi-level integration schemes in the context of generation gauge field samples.} use domain decomposition into blocks
%\worktodo{Tim: overlapping block -> Not true for pure gauge theory.}
to split the lattice into domains, write the action in terms of contributions from domains which finally leads to a factorization of the observable itself.
Each factor depends on a different set of gauge fields with support only on the domain of the decomposed lattice.
If applied appropriately, they lead to exponential signal-to-noise suppression in the prefactor of $e^{-(E_0 - E_{\text{noise}}/2)t}$, circumventing the Parisi-Lepage argument.
If applied to pure gauge theory this either completely removes the signal-to-noise problem or suppresses it substantially~\cite{Luscher:2001up,Meyer:2002cd,DellaMorte:2007zz,DellaMorte:2008jd,della2011novel}.
Once fermions enter the game, non-locality of the determinant and the fermion propagator are challenging to deal with in the factorization~\cite{Ce:2016idq,Ce:2016ajy,Giusti:2017ksp,Ce:2017ndt}.
In the meantime it has been successfully applied to connected and disconnected diagrams of meson two-point functions~\cite{Giusti:2018vxm} and the light-connected leading-order HVP of the muon $g-2$~\cite{DallaBrida:2020cik,Giusti:2021qhk}.

\tldr{multi-level is at gauge field sampling}
A multi-level integration scheme is a variance reduction method entering the gauge field generation phase.
Gauge fields in different domains are drawn according to the domain decomposed action.
This reduces the variance of the full translation averaged correlator, \ie the minimal variance. %what we will call the gauge variance or gauge noise of the observable in the following (the quantity will be formally defined in \cref{sec:numerics:gauge:variance}).
Reaching the gauge noise level with state-of-the-art estimators on large physical pion mass ensembles becomes prohibitively expensive, even more so if applied multi-level schemes reduce the gauge noise further.
In this work we are concerned with methods to reach the gauge noise with as little as possible computational resources, not with a reduction of the gauge noise itself\footnote{
In current high-precision determination of the LO-HVP, the level of the gauge variance is not the limiter of variance reduction, but rather the cost of proper translation averaging in the long distance.
}.
We thus expect the gauge fields to be generated already from now on.
Variance reduction methods discussed so far can thus be applied \emph{together} with a multi-level integration scheme to be even more efficient. 

\tldr{smearing methods}
%\worktodo{Tim: next 3 paragraphs: either remove completely or shrink and move below the disconnected diagrams which are much more relevant to the thesis}
In order to increase overlap with the true ground state of the system and reducing excited state contamination, investigation of different operators, with equal quantum numbers was put forward.
This led to smearing techniques for gauge and fermion fields.
%Instead of localized point sources which are unphysical, smeared localized sources were employed leading to smeared propagators.
%The goal is always an optimization of the signal.
Common smearing kernels are Jacobi~\cite{GUSKEN1989,PhysRevD.56.2743,Collins:1992fj,UKQCD:1993gym}, Wuppertal~\cite{GUSKEN1990361} or momentum smearing~\cite{Bali:2016lva}.
These techniques can be supplemented by smearing of gauge fields using APE~\cite{ALBANESE1987163}, HYP~\cite{Hasenfratz:2001hp} or Stout smearing~\cite{PhysRevD.69.054501}.

	% * Distillation: Distillation / Laplacian Heaviside smearing, eigenvectors of laplacian 
	% 	* was investigated as variance reduction method à la LMA in REF: lucius thesis
	% 	* original REF: https://journals.aps.org/prd/abstract/10.1103/PhysRevD.80.054506
	% 	* improved REF: https://inspirehep.net/literature/2087060

\tldr{distillation}
A noteworthy smearing method in the context of this thesis is distillation~\cite{HadronSpectrum:2009krc,Knechtli:2022bji}, where low modes of the 3D spatial gauge-covariant Laplacian are generated.
Source fields are then projected to that low-rank linear subspace; the projector acts as a smearing kernel.
%The method is based on the gauge-covariant smearing of fermion fields which in the limit exponentially suppress high mode contributions.
%Thus only the lowest mode components of the sources dominates ground state overlap.
Similar to LMA, the cost is proportional the number of modes determined and correlation functions restricted to that subspace can be volume averaged easily, since the restricted propagators are all-to-all.
The method was also investigated as a substitute for low-mode averaging~\cite{Bushnaq:2023} with no success.

\tldr{smearing changes amplitude}
Smearing of gauge and fermion fields has no effect on the asymptotic exponential behavior of the correlator but modifies its absolute value.
This makes smearing very suitable for spectroscopy studies, where one is interested in the exponential decay rate for mass extraction but unsuitable if contributions from excited states are required or the correlator amplitude is important too.
This is the case for the LO-HVP of the muon $g-2$.


\tldr{remainder of the part}
The remainder of this part is structured as follows.
\Cref{ch:p2:2pt-corr} introduces the main observable of interest -- the connected leading-order HVP of the muon $g-2$ -- and its form on the lattice.
This is followed by the state-of-the-art variance reduction method for this observable called low-mode averaging in \cref{ch:p2:lma} which this thesis is based upon.
Continuing with \cref{ch:p2:subspace-deflation}, we generalize the concept of low-mode averaging to arbitrary subspaces, whereas in \cref{ch:p2:lc} a method to approximately determine the low mode subspace used in solver preconditioning is shown.
These pieces are put together in \cref{ch:p2:multigrid} to construct a variance reduction method called \emph{multigrid low-mode averaging}, which we will show, solves the $V^{2}$-problem.
In \cref{ch:p2:numerics}, we show it in action applied to the LO-HVP and provide numerical evidence for its flat volume dependence.
This part is closed in \cref{ch:p2:chirality} with theoretical considerations about chirality.

%\worktodo{ ... and LMA is discused in ...}

