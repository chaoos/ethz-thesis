%*******************************************************
% Abstract
%*******************************************************
%\renewcommand{\abstractname}{Abstract}
\pdfbookmark[1]{Abstract}{Abstract}
\begingroup
\let\clearpage\relax
\let\cleardoublepage\relax
\let\cleardoublepage\relax

\chapter*{Abstract}

The \openqxd codebase was interfaced to \quda, enabling utilization of modern GPU-based compute clusters in the observable evaluation phase of computational lattice field theory, where solves of the Dirac equation can be offloaded to \quda.
Features like spatial \Cstar boundary conditions and the QCD+QED Wilson-Clover Dirac operator were implemented directly in \quda.
Special emphasis was taken on versatility and convenience on the application level for interaction with the interface allowing hybrid and heterogeneous workloads.
Further focus was drawn on efficient performance on a variety of modern compute infrastructure culminating in a range of scaling studies.
Code robustness can be guaranteed by a CI/CD pipeline running directly on production machines at CSCS.

A novel variance reduction method called multigrid low-mode averaging was developed and investigated.
The method is inspired by solver preconditioning methods such as multigrid and inexact deflation, where efficient low-mode subspaces are generated by exploiting local coherence of low modes.
Since traditional methods show disadvantageous quadratic scaling in the lattice volume making them prohibitively expensive as volumes increase, the numerical study focuses on the volume scaling behavior of the proposed method.
The main result of this thesis shows linear volume scaling of the new method, while achieving constant variance reduction with a constant number of low modes of \num{50} on all investigated lattices.
This renders the method a promising candidate for observables suffering from the signal-to-noise ratio problem on large physical point lattices near the chiral limit.

\endgroup

\cleardoublepage%

\begingroup
\let\clearpage\relax
\let\cleardoublepage\relax
\let\cleardoublepage\relax

\begin{otherlanguage}{ngerman}
\pdfbookmark[1]{Zusammenfassung}{Zusammenfassung}
\chapter*{Zusammenfassung}

Das Simulationsprogramm \openqxd wurde mit einem Interface zu \quda ausgestattet, womit moderne GPU-basierte Rechencluster für die Observablenberechnung in der computergestützten Gitterfeldtheorie genutzt werden können, insbesondere durch das Auslagern der Lösung der Diracgleichung an \quda.
Funkionalitäten sowie raumliche \Cstar Randbedingungen sowie der QCD+QED Wilson-Clover Dirac Operator wurden direkt in \quda implementiert.
Besonderer Wert wurde auf Vielseitigkeit und Anwenderfreundlichekeit gelegt, um eine flexible Interaktion mit der Schnittstelle und hybride oder heterogene Arbeitlasten zu gewährleisten.
Ein weiterer Schwerpunkt wurde auf die effiziente Nutzung von moderner Recheninfrastrukturen gelegt, was in einer Reihe von Skalierungsstudien resultierte.
Die Robustheit des Codes wird durch einen CI/CD Pipeline gewährleistet, die direkt auf Produktionsmaschinen am CSCS ausgeführt wird.

Eine neue Varianzreduktionsmethode names Multigrid Low-Mode Averaging wurde entwickelt und untersucht.
Die Methode ist inspiriert durch Vorkonditionierungstechniken wie Multigrid oder inexact Deflation, bei denen effizient Unterräume assoziiert zu niedrigen Moden mittels Ausnutzung lokaler Kohärenz konstruiert werden.
Da herkömmliche Verfahren eine ungünstige quadratische Skalierung im Gittervolumen aufweisen und bei grossen Volumina schnell undurchführbar werden, konzentriert sich die numerische Untersuchung auf das Volumen-Skalierungsverhalten der neuen Methode.
%Das Hauptergebnis dieser Arbeit zeigt, dass die neue Methode eine lineare Volumenskalierung erreicht und gleichzeitig die Varianzreduktion konstant halten kann bei einer konstanten Anzahl von \num{50} niegriger Moden auf allen untersuchten Gittern.
Das Hauptergebnis dieser Arbeit zeigt, dass die neue Methode eine lineare Volumenskalierung erreicht und gleichzeitig die Varianzreduktion bei einer konstanten Anzahl von \num{50} niedriger Moden auf allen untersuchten Gittern konstant halten kann.
Dies macht sie zu einem vielversprechendem Kandidaten für Observablen, die unter dem Signal-zu-Rausch-Verhältnis leiden, speziell auf grossen physikalischen Gittern nahe des chiralen Limits.

\end{otherlanguage}

\endgroup

\vfill